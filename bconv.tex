
\documentclass[12pt]{article}
\usepackage{amsmath, amsfonts, amsthm}
\usepackage[all]{xy}
\newtheorem{lemma}{Lemma}
\newtheorem{thm}{Theorem}
\newtheorem{prop}{Proposition}
\newcommand{\<}{\langle}
\def\>{\rangle}
\newcommand{\ct}[1]{\mathbf{#1}}



\begin{document}

\section{Ordered vector spaces and order unit spaces}

Let $V$ be a real vector space, $P\subseteq V$. Then $P$ is a convex cone if for any $a,b\in \mathbb R^+$, $x,y\in P$, we have $ax+by\in P$. A convex cone $P$ is pointed
 if $P\cap -P=\{0\}$ and generating if $V=P-P$.


\section{The categories $\ct{Conv}$ and $\ct{Bconv}$}


A general convex set, or a convex structure, was defined in \cite{gudder} as a set $X$ endowed with a ternary operation $\<\cdot,\cdot,\cdot\>: [0,1]\times X\times X\to X$,
 satisfying certain relations. For subsets in a vector space, we may put $\<s,x,y\>=sx+(1-s)y$ for $s\in [0,1]$ and $x,y\in X$, to obtain the usual definition of a convex set.  If $(X,\<\cdot,\cdot,\cdot\>_X)$ and $(Y,\<\cdot,\cdot,\cdot\>_Y)$ are convex structures, a map $f:X\to Y$ is called affine if it satisfies
\[
\<s,f(x),f(y)\>_Y=f(\<s,x,y\>_X),\qquad s\in [0,1],\ x,y\in X.
\]
The category of convex structures with affine maps as morphisms will be denoted by $\ct{Conv}$. 

For $X\in \ct{Conv}$, elements of $Hom(X,\mathbb R)$ are called functionals. Note that $Hom(X,\mathbb R)$ can be given a structure of a vector space, this will be denoted by $\widetilde A(X)$.  Let $A(X)$ be the vector subspace of bounded functionals, $A(X)^+$ the set of positive bounded functionals and let $1_X$ denote the constant $1_X(x)\equiv 1$. It is easy to see that $A(X)^+$ is a pointed convex cone and that $(A(X),A(X)^+,1_X)$ is an  order unit Banach space, with order unit norm satisfying
\[
\|f\|_{1_X}=\sup_{x\in X} |f(x)|.
\]
Let also $E(X):=Hom(X,[0,1])$, then $E(X)$ is the interval between 0 and $1_X$ in $(A(X),A(X)^+)$. Functionals in  $E(X)$ will be called effects.





A nonempty convex structure $X$ is geometric if it is isomorphic to a convex subset of a vector space. Any such isomorphism will be called a geometric representation of $X$.
Let $\phi:X\to V$ be a geometric representation and let $\tilde \phi:X\to V\oplus \mathbb R$ be defined by $\tilde \phi(x)=(\phi(x),1)$, then $\tilde \phi$ is a geometric 
representation  such that the image $\phi(X)$ lies in a hyperplane not containing 0. We may therefore assume without loss of generality  that any geometric representation has this property.

 Note that not all convex structures are geometric (Example?). We have the following characterization of geometric convex structures.
\begin{thm}(\cite[Thm. 2.2]{gudder}) A convex structure $X\neq \emptyset$ is geometric if and only if it is separated by elements of $\widetilde A(X)$, 
that is, for all $x\ne y\in X$, there is some functional $f\in \widetilde A(X)$ such that $f(x)\ne f(y)$. 
\end{thm}


Let $\phi:X\to V$ be a geometric representation and let $V(X):=\mathrm{span}\{\phi(X)\}$. Strictly speaking, this definition depends on the representation $\phi$, but all
 these  spaces  are isomorphic. We will also identify $x\equiv \phi(x)$ for all $x\in X$. By our assumption on the representation, there is a linear functional $\xi\in V(X)^*$ such that 
 $\xi(x)=1$ for all $x\in X$. Put $V(X)^+:=\cup_{\lambda\ge 0} \lambda X$. We now list some standard results.

\begin{lemma} %vynech%
$V(X)^+$ is a pointed and generating convex cone in $V(X)$, with base $X$.

\end{lemma}

\begin{proof} $V(X)^+$ is a generating convex cone in $V(X)$ by definition. Let $v\in V(X)^+\cap -V(X)^+$, so that there are some $a,b\in \mathbb R^+$ and $x,y\in X$ such that $v=ax=-by$. Assume $a+b>0$, then by convexity 
\[
0=\frac a{a+b}x+\frac b{a+b}y\in X,
\]
 which is impossible. Hence $a=b=0$ and $v=0$. To show that $X$ is a base of $V(X)^+$, it suffices to observe that $X=\{v\in V(X)^+, \xi(v)=1\}$.


\end{proof}



\begin{lemma} Any $f\in A(X)$ extends to a (unique) linear functional on $V(X)$, which is positive if and only if $f\in A(X)^+$.

\end{lemma}

\begin{proof} Put $f(0):=0$ and for $v=ax-by$, put $f(v):=ax-by$. We only need to prove that this extension is well defined. Assume that $v=ax-by=cx'-dy'$ for $a,b,c,d\in \mathbb R^+$ and $x,x',y,y'\in X$.


\end{proof}


\begin{lemma} Let $P$ be a pointed generating convex cone in a vector space $V$, with base $K$. For $v\in V$, put 
\[
\|v\|=\inf\{a+b,\ v=a x-b y,\ a,b\in \mathbb R^+, x,y\in K\}.
\]
This defines a seminorm in $V$, which is a norm if and only if $S:=co(K\cup -K)$ is linearly bounded. (linearly bounded means that the intersection of $S$ with any line is either empty or a bounded segment)

\end{lemma}

\begin{proof} Note that $\|\cdot\|$ is the Minkowski functional of $S$, that is
\[
\|v\|= \inf\{\lambda>0, v\in \lambda S\}.
\]
To see this, observe that $S=\{s x-(1-s)y,\ s\in [0,1],\ x,y\in K\}$. Denote the Minkowski functional by $p_S$. If $v=ax-by$ for some $a,b\in \mathbb R^+$ and $x,y\in K$, then if $a+b=0$, we must have 
$v=0$ and the equality obviously holds. Otherwise, 
\[
v=(a+b)(\frac a{a+b} x-\frac b{a+b} y)\in (a+b) S,
\]
so that $p_S(v)\le \|v\|$. On the other hand, let $v\in \lambda S$ for some $\lambda>0$. Then $v=\lambda(sx-(1-s)y)$ for some $s\in [0,1]$ and $x,y\in K$, so that 
\[
\|v\|\le \lambda s+(1-\lambda)(1-s)=\lambda,
\]
hence $\|v\|\le p_S(v)$. It is easy to see that $S$ is convex, balanced and absorbing in $V$, so that $p_S$ is a seminorm.  

Assume that $\|\cdot\|$ is a norm and let $v_t:=v+tw$ be a line in $V$. Then $\|w\|>0$ and $v_t\in  S$ implies that $1\ge \|v_t\|\ge |\|v\|-|t|\|w\||$, so that $|t|\le \tfrac{1+\|v\|}{\|w\|}$. Conversely, assume that $S$ is linearly bounded and let $\|v\|=0$. This implies $tv\in S$ for all $t\in \mathbb R$, hence we must have $v=0$. 

\end{proof}



The (semi)norm in the above lemma is called  the base (semi)norm in $V$. If $\|\cdot\|$ is a norm, the triple $(V,P,K)$ is called a base-normed space. We will denote 
by $\ct{BN}$ the category of base-normed spaces, with morphisms $f: (V_1,P_1,K_1)\to (V_2,P_2,K_2)$ given by linear maps $f:V_1\to V_2$ such that $f(X_1)\subseteq X_2$. It is easy to see that any morphism is a contraction with respect to the corresponding base norms.



Let $X$ be a geometric convex structure such that $co(X\cup -X)$ is linearly bounded in $V(X)$, then we will say that $X$ is bounded.
 The (sub)category of  bounded convex structures will be denoted by $\ct{Bconv}$. It is clear that $\ct{Bconv}$ and $\ct{BN}$ are equivalent categories. In particular,
for $X\in \ct{Bconv}$, $(V(X),V(X)^+,X)\in \ct{BN}$ and any morphism $f:X\to Y$ extends to a morphism of the corresponding base-normed spaces. 


We have the following characterization of objects in $\ct{Bconv}$.
\begin{prop} Let $X$ be a nonempty convex structure. Then the following conditions  are equivalent.
\begin{enumerate}
\item[(i)] $X$ is separated by $A(X)$.
\item[(ii)] $X$ is separated by $E(X)$.
\item[(iii)] $X$ is bounded.
\end{enumerate}

\end{prop}


\begin{proof} Note that since $E(X)$ contains an order unit, $A(X)$ is spanned by $E(X)$, so that (i) and (ii) are equivalent. Assume (i), then by
 Theorem \ref{thm:gudder}, $X$ is geometric. We will first show that any $f\in E(X)$ extends uniquely to a linear functional on $V(X)$.



Let $S=co(X\cup -X)$ and let $v_t:=v+tw$ for $v,w\in V(X)$, $w\ne 0$ and $t\in \mathbb R$. 
Note that there must be some $f\in E(X)$ such that $f(w)\ne0$. Indeed, we have $w=ax-by$ for $a,b\in \mathbb R^+$ and $x,y\in X$. If $f(w)=0$ for all $f\in E(X)$, 
 then also $a-b=1_X(w)=0$, hence $a=b$ and $w=a(x-y)$. 
  


\end{proof}


\begin{lemma} Let $X$ be a bounded convex structure. Then  $A(X)$ is the dual of $V(X)$ as a normed space, $A(X)^+$ is the dual cone to $V(X)^+$ and 
\[
X=\{v\in V(X)^+, 1_X(v)=1\}.
\]
\end{lemma}

\begin{proof} Let $V(X)^*$ denote the set of bounded linear functionals on $V(X)$ with the base norm. Clearly, any $\varphi\in V(X)^*$ restricts to an affine map 
$f_\varphi:X\to \mathbb R$ and since $\varphi$ is bounded, we have
\[
\sup_{x\in X} |f_\varphi(x)|\le \sup_{v\in co(X\cup -X)}|\varphi(v)|\le \sup_{\|v\|\le 1}|\varphi(v)|=\|\varphi\|,
\]
so that $f_\varphi\in A(X)$, with $\|f_\varphi\|\le \|\varphi\|$.


\end{proof}



\end{document}

