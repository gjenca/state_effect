
\documentclass[12pt]{article}
\usepackage{amsmath, amsfonts, amsthm, amssymb}
\usepackage[all]{xy}
\newtheorem{lemma}{Lemma}
\newtheorem{thm}{Theorem}
\newtheorem{prop}{Proposition}
\newtheorem{coro}{Corollary}

\theoremstyle{remark}
\newtheorem{rem}{Remark}
\newcommand{\<}{\langle}
\def\>{\rangle}
\newcommand{\ct}[1]{\mathbf{#1}}



\begin{document}

\section{Ordered vector spaces }

Overall reference:  \cite{jameson}

\subsection{Basic definitions}

Let $X$ be a real vector space. A subset $A\subseteq X$ is 
\begin{itemize}
\item algebraically open (closed) if the intersection of any line with $A$ is an open (closed) subset of the line
\item linearly bounded if the intersection of $A$ with any line is  a bounded subset of the line
\end{itemize}
We say that $a\in A$ is an algebraic interior point of $A$ if it is an interior point of the intersection of any line with $A$, that is, 
for any $x\in X$ there is some $\delta>0$ such that $a+sx\in A$ for all $|s|\le \delta$. 
 The set of all such points is called the algebraic interior of $A$ and is denoted by $aint(A)$. The algebraic closure of $A$ is $acl(A):=X\setminus aint(X\setminus A)$.
If $A$ is convex, then
\[
acl(A)=\{x\in X,\ \exists y\in X,\ x+\lambda y\in A,\ \forall \lambda\in (0,1)\}.
\]
$A$ is algebraically open iff $A=aint(A)$ and algebraically closed iff $A=acl(A)$. If $A$ is convex, then both $aint(A)$ and $acl(A)$ are convex as well.

\begin{rem} (cf. \cite[\S 16]{kothe}) 
If $A$ is convex, then $aint(A)$ is algebraically open, but  in general $aint(aint(A))\subsetneq aint(A)$. The algebraic closure is not necessarily algebraically closed even if $A$ is convex. But if $A$ is convex and $aint(A)\ne \emptyset$, then $acl(acl(A))=acl(A)$. \end{rem}

%For $A\subseteq X$, the associated Minkowski functional $\mu_A: X\to \mathbb R^+\cup\{\infty\}$ is defined as 
%\[
%\mu_A(x):=\inf\{\lambda>0, x\in \lambda A\},\qquad x\in X
%\]
%If $A$ is convex and $aint(A)\ne \emtyset$, $\mu_A$ is a seminorm in $X$. 

\subsubsection*{Wedges, cones and orderings}

A subset $P\subseteq X$ is called a  wedge if $P+P\subseteq P$ and $\lambda P\subseteq P$ for any $\lambda\ge 0$. The preorder $x\le y$ if $x-y\in P$ is compatible with the linear structure, such a preorder is called an ordering in $X$. Conversely, for any ordering, the set of positive elements  is a wedge.

The pair $(X,P)$ where $P$ is a wedge is called an ordered vector space. 
The corresponding ordering is a partial order iff $P\cap -P=\{0\}$, in this case $P$ is called a cone. 
$X$ with this ordering is directed  iff $P$ is generating, that is,  $X=P-P$. 

\subsubsection*{Positive maps}

Let $(X,P)$ and $(Y,Q)$ be ordered vector spaces. A  linear map $F:X\to Y$ is called positive if $F(P)\subseteq Q$.
 Let  $(P,Q)$ denote the set of positive maps, then $(P,Q)$ is a wedge in the vector space $L(X,Y)$ of all linear maps $X\to Y$. We have

\begin{lemma}\label{lemma:duality_morph} $(P,Q)$ is a cone if and only if $P$ is generating and $Q$ is a cone.

\end{lemma}

\subsubsection*{Archimedean and almost Archimedean orderings}

Let $(X,P)$ be an ordered vector space. We say that the ordering (or $P$) is Archimedean if 
 $x\le \lambda y$ for some $y\in X$ and all $\lambda> 0$ implies that $x\le 0$. 
\begin{prop}
The following are equivalent.
\begin{enumerate}
\item[(i)] the ordering is Archimedean.
\item[(ii)]  $\exists y\in X, \epsilon >0$ such that  $x\le \lambda y$ for all $\epsilon\ge \lambda>0$ $\implies$ $x\le 0$.
\item[(iii)] $\exists y\in P, \epsilon >0$ such that  $x\le \lambda y$ for all $\epsilon\ge \lambda>0$ $\implies$ $x\le 0$.
\item[(iv)] $P=acl(P)$.
\end{enumerate}
\end{prop}

The ordering (or $P$) is almost Archimedean if $-\lambda y\le x\le \lambda y$ for some $y\in X$ and  all $\lambda> 0$ implies  $x=0$.

\begin{prop}
The following are equivalent.
\begin{enumerate}
\item[(i)] the ordering is almost Archimedean.
\item[(ii)]  $\exists y\in X, \epsilon >0$ such that  $-\lambda y\le x\le \lambda y$ for all $\epsilon\ge \lambda>0$ $\implies$ $x= 0$.
\item[(iii)] $\exists y\in P, \epsilon >0$ such that  $-\lambda y\le x\le \lambda y$ for all $\epsilon\ge \lambda>0$ $\implies$ $x= 0$.
\item[(iv)] $acl(P)$ is a cone.
\end{enumerate}
\end{prop}

\begin{rem}
Note that an almost Archimedean wedge must be a cone. An Archimedean  wedge is almost Archimedean iff it is a cone.
\end{rem}

\subsection{Order units and bases}


\subsubsection*{Order units and seminorms}

 An element $u\in X$ is an order unit in $(X,P)$ if for any $x\in X$, there is some $\lambda\in \mathbb R^+$ such that 
 $x\le \lambda u$. This is equivalent to $u\in aint(P)$. If $aint(P)\ne \emptyset$, $P$ is generating.  

If $u$ is an order unit, then $P$ is (almost) Archimedean iff 
$u$ is (almost) Archimedean: $x\le \lambda u$ for all $\lambda>0$ implies $x\le 0$ (resp. $-\lambda u\le x\le \lambda u$ for all $\lambda>0$ implies $x=0$). 



For an order unit $u$, put
 \[
\|x\|_u=\inf\{\lambda>0,\ -\lambda u\le x\le \lambda u\}.
 \]
Then $\|\cdot\|_u$ is a seminorm in $X$. It is a norm iff  $u$ is almost Archimedean. 

\begin{rem}
If $u_1,u_2\in aint(P)$, the associated seminorms $\|\cdot\|_{u_1}$ and $\|\cdot\|_{u_2}$ are equivalent. The corresponding topology is thus a property of the ordering rather than the order unit. In fact, this topology is the finest locally convex topology 
 making all order intervals bounded.
\end{rem}
 

\begin{lemma} Let $u$ be Archimedean. Then $[-u,u]=\{x,\in X,\ \|x\|_u\le 1\}$ and the wedge $P$ is closed in the topology given by $\|\cdot\|_u$.

\end{lemma}

\begin{proof} Let $x\in [-u,u]$, then clearly $\|x\|_u\le 1$. Conversely, assume that $\|x\|_u\le 1$, then $-(1+\epsilon) u\le x\le (1+\epsilon) u$ for all $\epsilon>0$. 
This implies that $\pm x-u\le \epsilon u$ for all $\epsilon>0$ and since $u$ is Archimedean, this implies $\pm x\le u$, that is, $x\in [-u,u]$.

For the second statement, let $x\in \bar P$ (the closure of $P$ w.r. to $\|\cdot\|_u$). Then for all $n\in \mathbb N$, there is some $p_n\in P$ such that $\|x-p_n\|_u\le \tfrac1n$. This implies that $-x\le p_n-x\le \tfrac 1n u$ for all $n$ and since  $u$ is Archimedean, $-x\le 0$, so that $x\in P$.

\end{proof}

\subsubsection*{Order unit spaces}

A triple $(X,P,u)$ where $X$ is a vector space, $P\subseteq X$ an Archimedean  cone and $u\in aint(P)$ is called an order unit space. To summarize, in this case, 
$\|\cdot\|_u$ is a norm in $X$, $[-u,u]$ is the corresponding closed unit ball and $P$ is norm closed.
If $(X_i,P_i,u_i)$, $i=1,2$ are order unit spaces, a linear map $f:X_1\to X_2$ is called unital if $f(u_1)=u_2$.

\begin{prop} Let $(X_i,P_i,u_i)$, $i=1,2$ be order unit spaces. Any positive unital map $f:X_1\to X_2$ is a contraction.
\end{prop}

\subsubsection*{Bases and seminorms}

Let $(X,P)$ be an ordered vector space. A convex subset $K\subset P$ is called a base of $P$ if for any nonzero $p\in P$ there is a unique $\lambda>0$  such that 
$\lambda p\in K$.  


\begin{lemma} The wedge  $P$ has a base if and only if $P$ is a cone and there exists a linear functional $\xi$ on $X$ which is strictly positive on $P$. Then 
 $K=\{p\in P, \xi(p)=1\}$. 
\end{lemma}

\begin{proof} Let $K$ be a base of $P$, and let $0\neq x\in P\cap -P$. Then there are $\lambda,\mu >0$ such that $\lambda x=x_1\in K$ and $-\mu x=x_2\in K$.  
It follows that $\lambda^{-1}x_1=-\mu^{-1}x_2$ and then  $\tfrac{\mu}{\lambda+\mu}x_1+\tfrac{\lambda}{\lambda+\mu}x_2=0$. Since $K$ is convex, we obtain $0\in K$, but then 
 for any $p\in K$,  $\lambda p\in K$ for all $\lambda\in[0,1]$. Hence $P$ must be a cone.  For $p\in P$, let $\xi(p)$ be the unique positive number such that 
$\xi(p)^{-1}p\in K$. Then clearly $\xi(sp)=s\xi(p)$. Further, let $p,q\in P$ and let $\alpha=\xi(p)+\xi(q)$, then 
\[
\alpha^{-1}(p+q)=\frac{\xi(p)}{\alpha} \xi(p)^{-1}p+ \frac{\xi(q)}{\alpha} \xi(q)^{-1}q\in K,
\]
so that  $p\mapsto \xi(p)$ is an additive function  $\xi: P\to \mathbb R^+$. The function $\xi$ easily extends to $P-P$ and has an extension to all of $X$ by Hahn-Banach theorem. This extension is obviously positive and $K=\{p\in P,\ \xi(p)=1\}$.  

Conversely, let $\xi:X\to \mathbb R$ be strictly positive, then $K=\{p\in P, \xi(p)=1\}$ is a convex subset of $P$ and $\xi(p)^{-1}p\in K$ for any $p\in P$. Uniqueness is obvious. 



\end{proof}



\begin{prop} (\cite{ellis}) Let $P$ be a generating cone in a vector space $X$ and let $K$ be a base of $P$. For $x\in X$, put 
\[
\|x\|_K:=\inf\{a+b,\ x=a p-b q,\ a,b\in \mathbb R^+, p,q\in K\}.
\]
This defines a seminorm in $X$, which is a norm if and only if $S:=co(K\cup -K)$ is linearly bounded. 

\end{prop}

\begin{proof}
It can be checked easily that $\|\cdot\|_K$ is a seminorm. Note also that $x\in S$ implies $\|x\|_K\le 1$. Indeed, any $x\in S$ has the form $x=\lambda p-(1-\lambda)q$ for some  $\lambda\in [0,1]$, $p,q\in K$ and then  $\|x\|_K\le \lambda+(1-\lambda)=1$.  
Assume that $\|\cdot\|_K$ is a norm and let $x_t:=x+ty$ be a line in $X$. Then $\|y\|_K>0$ and $x_t\in  S$ implies that $1\ge \|x_t\|_K\ge |\|x\|_K-|t|\|y\|_K|$, so that $|t|\le \tfrac{1+\|x\|_K}{\|y\|_K}$. Conversely, assume that $S$ is linearly bounded and let $\|x\|_K=0$. This implies $tx\in S$ for all $t\in \mathbb R$, hence we must have 
$x=0$. 
 

\end{proof}

The (semi)norm in the above proposition  is called  the base (semi)norm in $X$. 

\begin{rem}
Note that $\|\cdot\|_K$ is the Minkowski functional of $S$, that is
\[
\|x\|_K= \inf\{\lambda>0, x\in \lambda S\}.
\]
To see this, observe that $S=\{s p-(1-s)q,\ s\in [0,1],\ p,q\in K\}$. Denote the Minkowski functional by $p_S$. If $x=ap-bq$ for some $a,b\in \mathbb R^+$ and $p,q\in K$, then if $a+b=0$, we must have 
$x=0$ and the equality obviously holds. Otherwise, 
\[
x=(a+b)(\frac a{a+b} p-\frac b{a+b} q)\in (a+b) S,
\]
so that $p_S(x)\le \|x\|_K$. On the other hand, let $x\in \lambda S$ for some $\lambda>0$. Then $x=\lambda(sp-(1-s)q)$ for some $s\in [0,1]$ and $p,q\in K$, so that 
\[
\|x\|_K\le \lambda s+(1-\lambda)(1-s)=\lambda,
\]
hence $\|x\|_K\le p_S(x)$. The set $S$ is convex, balanced and absorbing in $X$, so that $\|\cdot\|_K=p_S$ is a seminorm.  
\end{rem}




\begin{rem} Linear boundedness of $K$ is in general not enough. There are some weird infinite dimensional examples such that $K$ is linearly bounded but $co(K\cup -K)$ is not.

\end{rem}

\subsubsection*{Base-normed spaces}

A triple $(X,P,K)$, where $X$ is a vector space, $P$ a generating cone and $K$ a base of $P$ such that $co(K\cup-K)$ is linearly bounded is called a base-normed space. 
Let $(X_i,P_i,K_i)$ be base-normed spaces. A linear map $f:X_1\to X_2$ is called base-preserving if $f(K_1)\subset K_2$.


\begin{prop} Let $(X_i,P_i,K_i)$, $i=1,2$ be base-normed spaces. Any base-preserving linear  map $f:X_1\to X_2$ is a positive contraction.
\end{prop}

%\begin{prop}\label{prop:bn_complete}\cite{alfsen}  Let $(X,P,K)$ be a base-normed space and asume that $X$ carries a locally convex topology $\tau$ such that $K$ is $\tau$-compact.
%Then $X$ is $\|\cdot\|_K$-complete.

%\end{prop}
%\subsubsection*{Completeness}

%We give some sufficient conditions for completeness of order unit norms and base norms. 



\subsection{Duality}
\subsubsection*{The order dual}

Let $(X,P)$ be an ordered vector space and let $X'$ denote the algebraic dual of $X$. Then the dual wedge of $P$ is defined as
\[
P':=\{\varphi\in X', \varphi(p)\ge 0, \forall p\in P\}
\]
Then $(X',P')$ is an ordered vector space: the order dual of $X$. Note that  $P'=(P,\mathbb R^+)$ and it follows by  Lemma \ref{lemma:duality_morph} that 
	$P'$ is a cone iff $P$ is generating. Further, note that $p\in P\cap -P$ implies that $\varphi(p)=0$ for all $\varphi\in P'$, hence if $P'$ is generating, $P$ must be a cone. The converse is not true in general.


\subsubsection*{The norm dual of a vector space with an order unit norm} 


Let $(X,P)$ be an ordered vector space with an order unit $u$. Positive unital linear functionals are called states, the set of all 
 states will be denoted by $\mathcal S(X,P,u)$.

\begin{lemma}\label{lemma:separates}
 If  $\mathcal S(X,P,u)$ separates the points of $X$, then $\|\cdot\|_u$ is a norm.

\end{lemma}

\begin{proof}
Let $x\in X$, $-\lambda u\le x\le \lambda u$ and let $\varphi\in \mathcal S(X,P,u)$. Then $|\varphi(x)|\le \lambda$. It follows that 
$\sup_{\varphi\in \mathcal S(X,P,u)} |\varphi(x)|\le \|x\|_u$. Let $x\neq 0$ and let $\varphi\in  \mathcal S(X,P,u)$ be such that 
$\varphi(x)\ne 0$, then $0<|\varphi(x)|\le \|x\|_u$, so that $\|\cdot\|_u$ is a norm.


\end{proof}



Assume now that $\|\cdot \|_u$ is a norm. In this case,  $P$ is an almost Archimedean generating  cone. We do not assume that $(X,P,u)$ is an order unit space, so $u$ does not have to be Archimedean. 
Let $X^*$ be the normed space dual of $(X, \|\cdot\|_u)$ and let $\|\cdot\|_u^*$ be the norm in $X^*$. 

\begin{lemma} 
\begin{enumerate}
\item[(i)] Any $\varphi\in P'$ is bounded, with $\|\varphi\|_u^*=\varphi(u)$.
\item[(ii)] If $\varphi\in X^*$ is such that $\|\varphi\|_u^*=\varphi(u)$, then $\varphi\in P'$.
\end{enumerate}


\end{lemma}
\begin{proof}
 (i) is quite easy. For (ii), we may assume $\varphi(u)=1$. Let $x\in P$ and let $\lambda>0$ be such that $0\le x\le \lambda u$. Then
 $\|x-\lambda u\|_u\le \lambda$ and we have
 \[
|\varphi(x)-\lambda|=|\varphi(x-\lambda u)|\le \|\varphi\|_u^*\|x-\lambda u\|_u\le \lambda.
 \]
This implies $\varphi(x)\ge 0$.

\end{proof}



\begin{thm}\label{thm:ou_dual} Let $(X,P)$ be an ordered vector space wwith an order unit norm $\|\cdot\|_u$. Then 
$P'$  has a $w^*$-compact base $K$ such that  $(X^*,P',K)$ is a base-normed space and $\|\cdot\|_K=\|\cdot\|_u^*$. 

\end{thm}


\begin{proof}\cite{ellis} The set $K=\varphi\in \mathcal S(X,P,u)$  is a  $w^*$-compact  base of $P'$. We will show that the base seminorm $\|\cdot\|_K$ equals to the dual norm in $X^*$ and hence is itself a norm.

Let $Y=X\times X$ be ordered by the wedge $Q=P\times P$, then $(u,u)$ is an order unit in $(Y,Q)$.
Let 
\[
Z=\{t(u,u)-(x,-x),\ t\in \mathbb R, x\in X\},
\]
then $Z$ is a linear subspace in $Y$ containing the order unit. 
For  $\varphi\in X^*$,  put 
\[
F_\varphi(z)=t\|\varphi\|_u^*-\varphi(x), \qquad z=t(u,u)-(x,-x)\in Z
\]
This defines a linear functional on $Z$. Moreover, note that $z=t(u,u)-(x,-x)\in Q$ iff $\|x\|_u\le t$ and then $F_\varphi(z)\ge (t-\|x\|_u)\|\varphi\|_u^*\ge 0$.
Since $Z$ contains the order unit, $F_\varphi$ extends to a positive linear functional on $Y$ (e.g. Krein's theorem). Put 
\[
\psi_1(x)=F_\varphi(x,0),\quad \psi_2(x)=F_\varphi(0,x),\qquad x\in X.
\]
Then $\psi_1,\psi_2\in P'$ and $\varphi=\psi_2-\psi_1$, this shows that $P'$ is generating in $X^*$. Moreover, $F_\varphi(u,u)=\|\varphi\|_u^*$
\[
\|\varphi\|_u^*=F_\varphi(u,u)= \psi_1(u)+\psi_2(u)\ge \|\varphi\|_K
\]
On the other hand, let $\varphi=a\varphi_1- b\varphi_2$ with $a,b\ge 0$, $\varphi_1,\varphi_2\in K$, then
$\|\varphi\|_u\le a+b$, this shows the opposite inequality.

\end{proof}


\begin{coro} Let $(X,P)$ be an ordered vector space with an order unit $u$. Then $u$ is almost Archimedean iff $K=\mathcal S(X,P,u)$ separates the points of $X$.

\end{coro}

\begin{proof} Assume $u$ is almost Archimedean, then $\|\cdot\|_u$ is a norm. Let $X\ni x\ne 0$. By Theorem \ref{thm:ou_dual},
\[
\|x\|_u=\sup_{\|\varphi\|_u^*\le 1} |\varphi(x)|=\sup_{\varphi\in S} |\varphi(x)|=\sup_{\varphi\in K}|\varphi(x)|,
\]
so that we must have $\varphi(x)\ne 0$ for some state $\varphi$. The converse is Lemma \ref{lemma:separates}.

\end{proof}



\subsubsection*{The norm dual of a base-normed space}

Let $(X,P,K)$ be a base-normed space and let $X^*$ be the normed space dual of $(X,\|\cdot\|_K)$. Let $P^*= P'\cap X^*$.

\begin{thm} There is an order unit $u\in X^*$ such that $(X^*,P^*,u)$  is an order unit space.  

\end{thm}

\begin{proof} Let $(X,P,K)$ be a base-normed space. Note first that for any $\varphi\in X'$, we have
\[
\|\varphi\|_K^*=\sup_{x\in S} |\varphi(x)|=\sup_{x\in K} |\varphi(x)|,
\]
where $S=co(K\cup -K)$. There is a strictly positive functional $u\in X'$ such that 
$K=\{p\in P, u(p)=1\}$. Note that $u$ is a base-preserving linear map into the base-normed space $(\mathbb R,\mathbb R^+,1)$, hence 
is a positive contraction. Moreover, for  $\varphi\in X^*$ and $x\in K$, we have
$-\|\varphi\|_K\le \varphi(x)\le \|\varphi\|_K$, so that $-\|\varphi\|_Ku\le \varphi\le \|\varphi\|_Ku$, it follows that $u$ is an order unit in $(X^*, P'\cap X^*)$ and $\|\varphi\|_u\le \|\varphi\|_K^*$. Conversely,  $-\lambda u\le \varphi\le \lambda u$ 
implies that $\sup_{x\in K}|\varphi(x)|\le \lambda$, so that $\|\varphi\|_u= \|\varphi\|_K^*$. To show that $u$ is Archimedean, 
 let $\varphi\le \lambda u$ for all $\lambda>0$. Then for $x\in K$, $\varphi(x)\le \lambda$ for any $\lambda>0$, hence $\varphi(x)\le 0$. 

\end{proof}




\subsubsection*{Preduals}

We next discuss the Banach space preduals of order unit and base-normed spaces. Here $(X,\|\cdot\|)$ is a Banach space and $(X^*,\|\cdot\|^*)$ the dual space.  If $P\in X$ is a wedge, we will denote 
\[
P^*:=\{\varphi\in X^*,\ \varphi(p)\ge 0,\ \forall p\in P\}= P'\cap X^*.
\]
Similarly, if $Q$ is a wedge in $X^*$, we will denote 
\[
Q_*:=\{x\in X,\ q(x)\ge 0,\ \forall q\in Q\}= Q'\cap X.
\]
It is clear that $P^*$ and $Q_*$ are wedges. Moreover, $(P^*)_*=\bar P$ and $(Q_*)^*$ is the weak*-closure of $Q$. 

\begin{thm}\cite{ellis, asell} Let $X^*$ be  an order unit space with weak*-closed positive cone. Then $X$ is base-normed. More precisely, 
if there is an Archimedean weak*-closed cone $Q\subset X^*$ with an order unit $u$  such that $\|\cdot\|^*=\|\cdot\|_u$, 
 then  $Q_*\subset X$ has a base $K=\{p\in Q_*, u(p)=1\}$ and $(X,Q_*,K)$ is a base-normed 
space with $\|\cdot\|=\|\cdot\|_K$.

\end{thm}

\begin{proof}
Let $p\in Q_*$ be such that $u(p)=0$, then for any $\varphi\in Q$, 
\[
0\le \varphi(p)\le \|\varphi\|_u\varphi(u)=0.
\]
Since  $X^*=Q-Q$ separates points in $X$, we obtain $p=0$. Hence 
$u$ defines a strictly positive linear functional on $(X,Q_*)$  and $K$ is a base of $Q_*$. 
For $p\in Q_*$, we have
\[
\|p\|=\sup_{\varphi\in [-u,u]}|\varphi(p)|=u(p),
\]
it follows that $S=co(K\cup -K)$ is a subset of the unit ball of $X$. Hence $\|\cdot\|\le \|\cdot\|_K$ (since $\|\cdot\|_K$ is the Minkowski functional of $S$).   Since $Q=(Q_*)^*$, we have for $\varphi\in X^*$:
\begin{align*}
\|\varphi\|_u&=\inf\{\lambda >0,\ \lambda u\pm \varphi\in Q\}=\inf\{\lambda >0,\ (\lambda u\pm \varphi)(p)\ge 0,\ \forall p \in Q_*\}\\
&= \inf\{\lambda>0,\ |\varphi(p)|\le \lambda,\ \forall p\in K\}=\sup_{p\in K}|\varphi(p)|.
\end{align*}
Assume that $x_0\in X$ is such that $\|x_0\|\le 1$ and $x_0\ne \bar S$, then by Hahn-Banach separation theorem, there is some $\varphi\in X^*$ such that 
\[
\|\varphi\|_u=\sup_{p\in K}|\varphi(p)|=\sup_{x\in S}\varphi(x)< \varphi(x_0)\le \|\varphi\|^*=\|\varphi\|_u.
\]
It follows that $S$ is dense in the unit ball $X_1$ of $X$.  Choose any $\alpha>1$ and let $\alpha_n>0$ be a sequence such that  $1+\sum_n\alpha_n<\alpha$. There is some element $x_1\in S$ such that $\|x_0-x_1\|< \alpha_1$. Similarly, there is some $x_2\in \alpha_1S$ 
such that $\|x_0-x_1-x_2\|<\alpha_2$. Continuing by induction, we obtain a sequence $\{x_n\}$ in $X$ such that $\|x_n\|_K\le \alpha_{n-1}$ and  $\|x_0-\sum_n x_n\|<\alpha_n\to 0$. Hence  
\[
\|x_0\|_K=\|\sum_n x_n\|_K\le \sum_n\|x_n\|_K\le 1+\sum_n\alpha_n<\alpha,
\]
so that  $X_1\subset \alpha S$ and consequently $X=Q_*-Q_*$. Since the above inequality holds for all $\alpha>1$, we have
$\|\cdot\|=\|\cdot\|_K$.



%We will use  polar calculus. For $A\subset X$, let $A^\circ=\{\varphi\in X^*,\ \varphi(x)\le 1,\ \forall x\in A\}$. For $B\subset X^*$, $B^\circ$ is defined similarly. The following properties are shown easily: 
%\begin{enumerate}
%\item[(i)] $A^{\circ\circ}=\bar{co}(A\cup\{0\})$ ... etc
%\end{enumerate}
%




\end{proof}





\subsection{Categories of ordered vector spaces}


\begin{thebibliography}{99}
\bibitem{asell} L. Asimow and A. J. Ellis, \emph{ Convexity Theory and its Applications in Functional Analysis}, Academic Press, London, 1980
\bibitem{ellis} A. J. Ellis, The duality of partially ordered normed linear spaces, J. London Math. Soc. \textbf{39} (1964), 730-744
\bibitem{jameson} 
G. Jameson, \emph{Ordered Linear Spaces}, Lecture Notes in Mathematics, Springer, 1970
\bibitem{kothe} G. K\" othe, \emph{Topological Vector Spaces}, Springer, 1983
\end{thebibliography}


\end{document}

We will denote 
by $\ct{BN}$ the category of base-normed spaces, with morphisms $f: (X_1,P_1,K_1)\to (X_2,P_2,K_2)$ given by linear maps $f:X_1\to X_2$ such that $f(K_1)\subseteq K_2$. It is easy to see that any morphism is a contraction with respect to the corresponding base norms.

Let $(X,P,K)\in \ct{BN}$ and let $X^*$ be the dual space of linear functionals that are bounded with respect to the base norm. Let 
\[
P^*:=\{ \varphi\in X^*,\ \varphi(p)\ge 0,\ \forall p\in P\}
\]
and let $u$ be the functional such that $K=\{p\in P,\ u(p)=1\}$. Then $P^*$ is clearly a  cone.
Since $(\mathbb R, \mathbb R^+, 1)$ is a base-normed space with the usual norm in $\mathbb R$ as the base norm, $u$ is a morphism in $\ct{BN}$ and hence it is bounded. Moreover, it is easy to see that  $\varphi\in V^*$ satisfies
\[
\varphi(ay)=a\varphi(y)\le a\sup_{x\in K}|\varphi(x)|=\|\varphi\|u(ay),\quad \forall  a>0, y\in K.
\]
It follows that $u$ is an order unit in $(X^*,P^*)$, with $\|\varphi\|_u=\|\varphi\|$. Let $n\varphi\le u$ for all $n$, then for all $x\in K$, we have $\varphi(x)\le n^{-1}$, so that $\varphi\le 0$. We see that $(X^*,P^*,u)$ is an order unit space which is complete in the order unit norm, hence an order unit Banach space.


\section{The categories $\ct{Conv}$ and $\ct{Bconv}$}


A general convex set, or a convex structure, was defined in \cite{gudder} as a set $K$ endowed with a ternary operation $\<\cdot,\cdot,\cdot\>: [0,1]\times K\times K\to K$,
 satisfying certain relations. For subsets in a vector space, we may put $\<s,x,y\>=sx+(1-s)y$ for $s\in [0,1]$ and $x,y\in K$, to obtain the usual definition of a convex set.  If $(K,\<\cdot,\cdot,\cdot\>_K)$ and $(Y,\<\cdot,\cdot,\cdot\>_Y)$ are convex structures, a map $f:K\to Y$ is called affine if it satisfies
\[
\<s,f(x),f(y)\>_Y=f(\<s,x,y\>_K),\qquad s\in [0,1],\ x,y\in K.
\]
The category of convex structures with affine maps as morphisms will be denoted by $\ct{Conv}$. 

For $K\in \ct{Conv}$, elements of $Hom(K,\mathbb R)$ are called functionals. Note that $Hom(K,\mathbb R)$ can be given a structure of a vector space, this will be denoted by $\widetilde A(K)$.  Let $A(K)$ be the vector subspace of bounded functionals, $A(K)^+$ the set of positive bounded functionals and let $1_K$ denote the constant $1_K(x)\equiv 1$. It is easy to see that $A(K)^+$ is a pointed convex cone and that $(A(K),A(K)^+,1_K)$ is an  order unit Banach space, with order unit norm satisfying
\[
\|f\|_{1_K}=\sup_{x\in K} |f(x)|.
\]
Let also $E(K):=Hom(K,[0,1])$, then $E(K)$ is the interval between 0 and $1_K$ in $(A(K),A(K)^+)$. Functionals in  $E(K)$ will be called effects.





A nonempty convex structure $K$ is geometric if it is isomorphic to a convex subset of a vector space. Any such isomorphism will be called a geometric representation of $K$.
Let $\phi:K\to V$ be a geometric representation and let $\tilde \phi:K\to V\oplus \mathbb R$ be defined by $\tilde \phi(x)=(\phi(x),1)$, then $\tilde \phi$ is a geometric 
representation  such that the image $\tilde \phi(K)$ lies in a hyperplane not containing 0. We may therefore assume without loss of generality  that any geometric representation has this property.

 Note that not all convex structures are geometric (Example?). We have the following characterization of geometric convex structures.
\begin{thm}(\cite[Thm. 2.2]{gudder}) \label{thm:gudder} A convex structure $K\neq \emptyset$ is geometric if and only if it is separated by elements of $\widetilde A(K)$, 
that is, for all $x\ne y\in K$, there is some functional $f\in \widetilde A(K)$ such that $f(x)\ne f(y)$. 
\end{thm}


Let $\phi:K\to V$ be a geometric representation and let $V(K):=\mathrm{span}\{\phi(K)\}$. Strictly speaking, this definition depends on the representation $\phi$, but all
 these  spaces  are isomorphic. We will also identify $x\equiv \phi(x)$ for all $x\in K$. By our assumption on the representation, there is a linear functional $1_K\in V(K)^*$ such that 
 $1_K(x)=1$ for all $x\in K$. Put $V(K)^+:=\cup_{\lambda\ge 0} \lambda K$. We now list some standard results.

\begin{lemma} %vynech%
$V(K)^+$ is a pointed and generating convex cone in $V(K)$, with base $K$.

\end{lemma}

\begin{proof} $V(K)^+$ is a generating convex cone in $V(K)$ by definition. Let $v\in V(K)^+\cap -V(K)^+$, so that there are some $a,b\in \mathbb R^+$ and $x,y\in K$ such that $v=ax=-by$. Assume $a+b>0$, then by convexity 
\[
0=\frac a{a+b}x+\frac b{a+b}y\in K,
\]
 which is impossible. Hence $a=b=0$ and $v=0$. To show that $K$ is a base of $V(K)^+$, it suffices to observe that $K=\{v\in V(K)^+, 1_K(v)=1\}$.


\end{proof}



\begin{lemma}\label{lemma:extension} Any $f\in A(K)$ extends to a (unique) linear functional on $V(K)$, which is positive if and only if $f\in A(K)^+$.

\end{lemma}

\begin{proof} Put $f(0):=0$ and for $v=ax-by$, put $f(v):=ax-by$. We only need to prove that this extension is well defined, all the other  statements are obvious. Assume that $v=ax-by=cx'-dy'$ for $a,b,c,d\in \mathbb R^+$ and $x,x',y,y'\in K$. Then $ax+dy'=cx'+by$ and applying $1_K$ implies that $a+d=c+b$.
 If $a+d=0$, then $v=0$ and $f(v)=0=af(x)-bf(y)$. Otherwise, we obtain 
 \[
\frac a{a+d}x+ \frac d{a+d}y'=\frac c{c+b} x'+\frac b{c+b} y
 \]
and since $f$ is affine, we get back to $af(x)-bf(y)=cf(x')-df(y')$. 
\end{proof}






Let $K$ be a geometric convex structure such that $co(K\cup -K)$ is linearly bounded in $V(K)$, then we will say that $K$ is bounded.
 The (sub)category of  bounded convex structures will be denoted by $\ct{Bconv}$. It is clear that $\ct{Bconv}$ and $\ct{BN}$ are equivalent categories. In particular,
for $K\in \ct{Bconv}$, $(V(K),V(K)^+,K)\in \ct{BN}$ and any morphism $f:K\to Y$ extends to a morphism of the corresponding base-normed spaces. 


We have the following characterization of objects in $\ct{Bconv}$.
\begin{prop}\label{prop:bounded} Let $K$ be a nonempty convex structure. Then the following conditions  are equivalent.
\begin{enumerate}
\item[(i)] $K$ is separated by $A(K)$.
\item[(ii)] $K$ is separated by $E(K)$.
\item[(iii)] $K$ is bounded.
\end{enumerate}

\end{prop}


\begin{proof} Note that since $E(K)$ contains an order unit, $A(K)$ is spanned by $E(K)$, so that (i) and (ii) are equivalent. Assume 
(i), then by Theorem \ref{thm:gudder}, $K$ is geometric. By Lemma \ref{lemma:extension}, any $f\in E(K)$ extends uniquely to a linear functional on $V(K)$. 

Let $S=co(K\cup -K)$ and let $v_t:=v+tw$ for $v,w\in V(K)$, $w\ne 0$ and $t\in \mathbb R$. 
Note that there must be some $f\in E(K)$ such that $f(w)\ne0$. Indeed, we have $w=ax-by$ for $a,b\in \mathbb R^+$ and $x,y\in K$. If $f(w)=0$ for all $f\in E(K)$, 
 then also $a-b=1_K(w)=0$, hence $a=b$ and $w=a(x-y)$. It follows that either $a=0$ or $x=y$, but then $w=0$. If $v_t\in S$, 
  then 
  \[
  f(v_t)=f(v)+tf(w)\in f(S)=co(f(K)\cup -f(K))\subseteq [-1,1],
  \]
   and since $f(w)\ne 0$, this implies that $t$ must be in a bounded interval. Hence (iii) holds. 

Finally, if (iii) is true, then $(V(K), V(K)^+, K)$ is a base-normed space. Let $V(K)^*$ denote the normed space dual. 
Clearly, any $\varphi\in V(K)^*$ restricts to an affine map 
$f_\varphi:K\to \mathbb R$ and  we have
\begin{equation}\label{eq:dualnorm}
\sup_{x\in K} |f_\varphi(x)|\le \sup_{v\in co(K\cup -K)}|\varphi(v)|\le \sup_{\|v\|\le 1}|\varphi(v)|=\|\varphi\|,
\end{equation}
so that $f_\varphi\in A(K)$. Since the elements of $V(K)^*$ separate points of $V(K)$, this implies (i).


\end{proof}


\begin{lemma} Let $K$ be a bounded convex structure. Then  $A(K)$ is isometrically isomorphic to the dual of $V(K)$ as a normed space,  the dual cone to $V(K)^+$ is given by $A(K)^+$ and 
\[
K=\{v\in V(K)^+, 1_K(v)=1\}.
\]
\end{lemma}

\begin{proof} By \eqref{eq:dualnorm}, any $\varphi\in V(K)^*$ restricts to an  element  
 $f_\varphi\in A(K)$, with $\|f_\varphi\|\le \|\varphi\|$.  Conversely, by Lemma \ref{lemma:extension}, any $f\in A(K)$ extends to a unique linear functional $\varphi_f$ on $V(K)$
and clearly $\varphi_{f_\varphi}=\varphi$, $f_{\varphi_f}=f$. Let $v=ax-by$, then
\[
|\varphi_f(v)|=|af(x)-bf(y)|\le a|f(x)|+b|f(y)|\le (a+b)\|f\|,
\]
it follows that $|\varphi_f(v)|\le \|v\|\|f\|$, so that $\|\varphi_f\|\le \|f\|$. The rest is easy.


\end{proof}



\end{document}

