
\documentclass[12pt,letterpaper]{article}
\usepackage{amsmath,amsfonts,amsthm,amssymb}
\unitlength=1mm  %%%%%%%%%%%%%%%%%%%%%%%%%%%%%%%%%%%%%%%%%%%%%%% kvoli obrazku !!!
\renewcommand{\thesection}{\arabic{section}}
\hyphenation{ar-chi-me-de-an} \setlength\overfullrule{5pt}
%marks overlong lines in dvi file % Declaration section \swapnumbers
%Theorems, lemmas, etc. numbered at the left.
\newtheorem{lemma}{Lemma}[section]
\newtheorem{corollary}[lemma]{Corollary}
\newtheorem{theorem}[lemma]{Theorem}
\newtheorem{proposition}[lemma]{Proposition}
\newtheorem{definition}[lemma]{Definition}
\newtheorem{example}[lemma]{Example}
\newtheorem{remark}[lemma]{Remark}
\newtheorem{assumption}[lemma]{Standing Assumption}
\newcommand{\Nat}{{\mathbb N}} \newcommand\integers{{\mathbb Z}}
\newcommand\rationals{{\mathbb Q}} \newcommand\reals{{\mathbb R}}
\newcommand\complex{{\mathbb C}}
%Label items in an enumeration with small Roman
%numerals enclosed in parentheses.
\renewcommand{\theenumi}{{\rm(\roman{enumi})}}
\renewcommand\labelenumi{\theenumi} \newcommand\hilb{{\mathfrak H}}
\newcommand{\sgn}{\operatorname{sgn}} \newcommand{\dg}{\sp{\text{\rm
o}}}
\begin{document}
\title{Convex sets }
%\author{SP}
%\date{}
\maketitle


\section{Abstract convex structures }

Axioms for convex sets were introduced by H.M. Stone \cite{S}, and  then studied e.g. in \cite{G,CF}. We mostly follow \cite{G}.

\begin{definition}\label{de:conv} A convex structure is a set $X$ and a family of binary operations $<\lambda,x,y>$, $\lambda\in [0,1]$, on $X$ such that
\begin{enumerate}
\item[(c1)] $<\lambda, p,q> =<1-\lambda,q,p>$ (commutativity),
\item[(c2)] $<0,p,q>=p$ (endpoint condition),
\item[(c3)]  $<\lambda,p,<\mu,q,r>>=<\lambda\mu,<\nu,p,q>,r>$ ($\lambda\mu\neq 1$), where $\nu=\lambda(1-\mu)(1-\lambda\mu)^{-1}$ (associativity).
\end{enumerate}
\end{definition}

If $X$ is a convex subset of a real linear space, then $<\lambda,p,q>$ corresponds to $(1-\lambda)p+\lambda q$.
In \cite{S} and \cite{CF}, also the following axiom is given:

\begin{enumerate}
\item[(c4)] $<\lambda,p,p>=p$ (idempotence).
\end{enumerate}

A convex \emph{prestructure} is a set $S$ with a map $T:[0,1]\times S\times S \to S$ denoted $T(\lambda,p,q)=<\lambda,p,q>$ (no requirements on $T$).
An \rm{affine functional} is an affine map $f$ from a convex prestucture to the real line $\mathbb R$, that is, $f(<\lambda,p,q>)=(1-\lambda)f(p)+\lambda f(q)$ for all $\lambda \in [0,1]$ and $p,q\in S$. We denote by $S^*$ the set of all affine functional on $S$ and say that $S^*$ is \emph{total} (\emph{separating}) if for $p\neq q\in S$ there is $f\in S^*$ such that $f(p)\neq f(q)$.

\begin{theorem} A convex prestructure $S$ is isomorphic to a convex set iff $S^*$ is total.
\end{theorem}

\begin{proof} Suppose $S_0$ is a convex set and $F:S\to S_0$ an isomorphism. If $S_0$ is a convex subset of the linear space $V$, it is well-known that $V^*$ (the algebraic dual)
is total over $V$. Restricting the elements of $V^*$ to $S_0$, we get a total set of affine functionals for $S_0$. If $f\in V^*$, then $f\circ F\in S^*$, so $S^*$ is total.

Conversely, suppose that $S^*$ is total. For $p\in S$ define $J(p):S^*\to {\mathbb R}$ by $J(p)f=f(p)$. Clearly $S^*$ is a linear space under pointwise operations, and $J(p)\in S^{**}$ so that $J(S)\subseteq S^{**}$. Now $J(S)$ is a convex set - indeed, $(1-\lambda )J(p)+\lambda J(q)=<\lambda, p,q>$ -  and $J:S \to S^{**}$ is injective iff $S^*$ is total. Indeed, if $S^*$

is total and $p\neq q \in S$, then there is $f\in S^*$ with $f(p)\ne f(q)$ so $J(p)\neq J(q)$, and conversely, if $J$ is injective and $p\neq q\in S$,  then $J(p)\neq J(q)$ so that there is an $f\in S^*$ such that $f(p)=J(p)f\neq J(q)f= f(q)$. If follows that $J:S \to J(S)$ is an isomorphism.
\end{proof}

The following theorem gives an intrinsic characterization of those convex structures that are convex subsets of a linear space.

\begin{theorem}\label{th:intrinsic} \cite{S,CF}. A convex structure $S_1$  satisfying axioms \rm{(c1),(c2), (c3)} and in addition \rm{(c4)} embeds into a real vector space
iff the following cancellation property holds:
\begin{enumerate}
\item[(c5)] $<\lambda,x,y>=<\lambda,x,z>$ with $\lambda \in (0,1) \ \implies \ y=z$.
\end{enumerate}
\end{theorem}

\begin{proof} Clearly, every convex subset of a vector space satisfies this cancellation property.

Let $X$ be a convex structure satisfying (c4) and (c5). Let $V_X$ be the real vector space generated by $X$, so that $V_X$ has a base $(e_x)_{x\in X}$.
Let $U_X\subseteq V_X$ be a subspace generated by the vectors of the form
\[
 e_{<\lambda,x,y>}-(1-\lambda)e_x-\lambda e_y,  x,y\in X, \lambda \in [0,1].
\]

Let $W_X=V_X/U_X$ and let $\tilde{e_x}$ be the image of $e_x$. Then the mapping  $X \to W_X$, $x\mapsto \tilde{e_x}$ preserves convex combinations.
Vectors in $U_X$ have the form
\[
\sum_{i=1}^m \alpha_i(e_{<\lambda_i,a_i,b_i>}-(1-\lambda_i)e_{a_i}-\lambda_ie_{b_i})-\sum_{i=1}^m\beta_ie_{<\mu_i,c_i,d_i>}-(1-\mu_i)e_{c_i}-\mu_ie_{d_i})
\]
with  $\alpha_i,\beta_i\geq 0$, and $a_i,b_i,c_i,d_i\in X$ and $\lambda_i,\mu_i\in[0,1]$.
We split this into positive and negative terms as follows:
\begin{eqnarray*}
&\sum_{i=1}^m(\alpha_ie_{<\lambda_i,a_i,b_i>}+\beta_i(1-\mu_i)e_{c_i} +\beta_i\mu_ie_{d_i})\\  &-\sum_{i=1}^m(\beta_ie_{<\mu_i,c_i,d_i>}+\alpha_i(1-\lambda_i)e_{a_i}+\alpha_i\lambda_ie_{b_i})
\end{eqnarray*}
and observe that the sum of the coefficients of all negative terms equals to the sum of coefficients of all positive terms,
namely $\sum_i(\alpha_i+\beta_i)$. Without loss of generality we may assume  this sum to be $1$, then both the sums are convex combinations.  Interpreting these as convex combinations in $X$, these sums moreover define the same point in $X$.

We show the injectivity by proving that $\tilde{e_x}=\tilde{e_y}$ implies $x=y$, $x,y\in X$. The equation $\tilde{e_x}=\tilde{e_y}$ holds whenever $e_x-e_y$ lies in $U_X$.
If this is the case, then the first sum contains the term $\kappa e_x$, $\kappa >0$, and the second sum contains the term $\kappa e_y$ for the same $\kappa$, and all other terms cancel. Then both the sums define convex combinations of the same points with the same weights, except that the first contains the point $x$ with weight $\kappa$, while the second contains the point $y$ with weight $\kappa$. Applying the cancellation property, we obtain $x=y$.
\end{proof}


\section{Intrinsic metrics}

Let $S_1$ be a convex structure. For $p,q\in S_1$ define
\[
\sigma(p,q):=\inf\{ 0\leq \lambda\leq 1:<\lambda,p,p_1>=<\lambda,q,q_1>, p_1,q_1\in S_1\}
\]
Since $<\frac{1}{2},p,q>=<\frac{1}{2},q,p>$, we have $0\leq \sigma(p,q)\leq \frac{1}{2}$.
\[
\rho(p,q)=\frac{\sigma(p,q)}{1-\sigma(p,q)}, \ \  \mbox{then}  \ 0\leq \rho(p,q)\leq 1.
\]

\begin{theorem}\label{sigmarho} (\cite{G}) On any convex structure  $S_1$, $\sigma$ and $\rho$ are semimetrics.
\end{theorem}
\begin{proof} Clearly, $\sigma$ and $\rho$ are nonnegative and symmetric. We have to prove triangle inequality.
If $p=s$ or $q=s$, then $\sigma(p,q)\leq \sigma(p,s)+\sigma(s,q)$. Assume $p\neq s, q\neq s$.
Assume

$\lambda_1\in \{ 0< \lambda < 1: <\lambda,p,p_1>=<\lambda,s,s_1>, p_1,s_1\in S_1\}$;

$\lambda_2\in\{0 < \lambda < 1:<\lambda,s,s_2>=<\lambda,q,q_1> s_2,q_1\in S_1\}$,

$\lambda_3:=\lambda_1+\lambda_2-2\lambda_1\lambda_2$;

$p_2:=<\lambda_2(1-\lambda_1)\lambda_3^{-1},p_1,s_2>$;

$q_2:=<\lambda_2(1-\lambda_1)\lambda_3^{-1},s_1,q_1>$;

$\lambda_0:=\lambda_3(1-\lambda_1\lambda_2)^{-1}$.
Then

\begin{eqnarray*} <\lambda_0,p,p_2> &=& <\lambda_3(1-\lambda_1\lambda_2)^{-1},p,<\lambda_2(1-\lambda_1)\lambda_3^{-1},p_1,s_2>>\\
&=&<\lambda_2(1-\lambda_1(1-\lambda_1\lambda_2)^{-1},<\lambda_1,p,p_1>,s_2>\\
&=&<\lambda_2(1-\lambda_1)(1-\lambda_1\lambda_2)^{-1},<\lambda_1,s,s_1>,s_2>\\
&=&<(1-\lambda_2)(1-\lambda_1\lambda_2)^{-1},s_2,<\lambda_1,s,s_1>>\\
&=&<\lambda_1(1-\lambda_2)(1-\lambda_1\lambda_2)^{-1},<1-\lambda_2,s_2,s>s_1>\\
&=&<(1-\lambda_1)(1-\lambda_1\lambda_2)^{-1},s_1,<\lambda_2,q,q_1>>\\
&=&<(1-\lambda_1)(1-\lambda_2)(1-\lambda_1\lambda_2)^{-1},<\lambda_2(1-\lambda_1)\lambda_3^{-1},s_1,q_1>,q>\\
&=&<\lambda_3(1-\lambda_1\lambda_2)^{1},q,q_2>\\
&=&<\lambda_0,q,q_2>
\end{eqnarray*}
so $\lambda_0\in \{ 0< \lambda <1: <\lambda,p,p_2>=<\lambda,q,q_2>, p_2,q_2\in S_1\}$. Now since
\[
\lambda_0(1-\lambda_0)^{-1}=\lambda_1(1-\lambda_1)^{-1}+\lambda_2(1-\lambda_2)^{-1},
\]
we get
\begin{eqnarray*}
\rho(p,q)&=&\sigma(p,q)[1-\sigma(p,q)]^{-1}\leq \sigma(p,s)[1-\sigma(p,s)]^{-1}+\sigma(s,q)[1-\sigma(s,q)]^{-1}\\
&=&\rho(p,s)+\rho(s,q).
\end{eqnarray*}

The triangle inequality for $\sigma$ follows similarly from $\lambda_0\leq \lambda_1+\lambda_2$.

\end{proof}


\begin{theorem}\label{th:metric}  A necessary and sufficient condition for $\rho, \sigma$ to be metrics is that whenever there are sequences $\lambda_i\in [0,1]$, $p_i,q_i\in S_1$which satisfy $\lim_{i\to\infty} \lambda_i=0$, $<\lambda_i,p,p_i>=<\lambda_i,q,q_i>$ then $p=q$.
\end{theorem}

\begin{proof} Clearly $\rho$ is a metric iff $\sigma$ is. If $\sigma$ is a metric, since $\sigma(p,q)\leq \lambda_i$ $\forall i$ we have $p=q$. Conversely if $\sigma(p,q)=0$ then $V=\{ 0\leq \lambda_i\leq 1:<\lambda,p,p_1>=<\lambda,q,q_1>, p_1,q_1\in S_1\}$ either contains $0$ or $0$ is a limit point. In the former case $p=<0,p,p_1>=<0,q,q_1>=q$. In the latter case there exist $\lambda_i\in V$ with $\lambda_i \to 0$ so again $p=q$.
\end{proof}

\begin{corollary}\label{co:metric} Let $S_0$ be a convex set in a real vector space $X$. If there is a topology on $X$ that makes $X$ a Hausdorff topological vector space
in which $S_0$ is bounded, then $\rho$ is a metric.
\end{corollary}

\begin{proof} Suppose there are sequences $\lambda_i\in [0,1]$, $\lim \lambda_i=0, p_i,q_i\in S_0$ such that $(1-\lambda_i)p+\lambda_ip_i=(1-\lambda_i)q+\lambda_iq_i$. Then $p-q=\lambda_i(p-q)+\lambda_i(q_i-p_i)$. Let $\Lambda$ be a neighborhood of $0$. Then there is a neighborhood $W$ of $0$ such that $W+W+W\subseteq \Lambda$. Now for sufficiently large $i$, $\lambda_i(p-q)\in W$. Since $S_0$ is bounded, there is $\mu >0$ such that $\lambda S_0\subseteq W$ for $|\lambda|\leq \mu$. For $i$ sufficiently large, $\lambda_iq_i-\lambda_ip_i\in W+W$. Hence $p-q\in W+W+W\subseteq \Lambda$ for $i$ sufficiently large, and since $X$ is Hausdorff, $p-q=0$.
\end{proof}


The converse holds only in finite dimensional spaces.

Let $S_0$ be a convex set in a real linear space $V$, $\rho$ the intrinsic metric on $S_0$. Some terminology: $S_0$  is


$\bullet$  \emph{absorbing} iff  $\forall x\in V \ \exists\  \delta(x)>0$: $\lambda x\in S_0$ $\forall \lambda$ with $|\lambda|\leq \delta(x)$.

$\bullet$  \emph{balanced} iff $x\in S_0$, $|\lambda |\leq 1$ $\implies$ $\lambda x\in S_0$.

$\bullet$ \emph{radial} iff $x\in S_0$, $0\leq \lambda \leq 1$ $\implies$ $\lambda x\in S_0$.

$\bullet$ \emph{normalized} iff $\in S_0$, $\alpha \neq 1$ $\implies$ $\alpha x\notin S_0$.

Define $P:=\{ \alpha S_0: \alpha \geq 0\}$, then $P$ is a wedge.


\begin{definition}\label{de: basenorm} $x\in X$, $\|x\|:=\inf\{ c+d: x=cp-dq; c,d\geq 0; p,q\in S_0\}$.
\end{definition}

\begin{theorem} If $S_0$ is normalized or radial then $\|p-q\|=2\rho(p,q)$, $p,q\in S_0$. Moreover, $\|.\|$ is a norm iff $\rho$ is a metric.
\end{theorem}

\begin{proof} Assume $S_0$ is normalized. For $p,q\in S_0$, if $p-q=cp_1-dq_1$, $p_1,q_1\in S_0$, $c,d\geq 0$, then $p+dq_1=q+cp_1$. which implies
\[
(1+d)(\frac{p}{1+d}+\frac{dq_1}{1+d})=(1+c)(\frac{1}{1+c}q+\frac{c}{1+c}p_1).
\]
Then $q_2:=\frac{1}{1+c}q+\frac{c}{1+c}p_1\in S_0$, and  $\frac{1+c}{1+d}q_2=\frac{p}{1+d}+\frac{dq_1}{1+d}\in S_0$  $\implies$ $\frac{1+c}{1+d}=1$ $\implies$ $c=d$.

So all representations of $p-q$ are of the form $c(p_1-q_1), c\geq 0, p_1,q_1\in S_0$. We then have
\begin{eqnarray*} \sigma(p,q)&=&\inf \{ 0\leq \lambda <1: (1-\lambda)p+\lambda p_1=(1-\lambda)q+\lambda q_1, p_1,q_1\in S_0\}\\
&=&\inf\{ 0\leq \lambda <1: p-q=\frac{\lambda}{1-\lambda}(q_1-p_1),p_1,q_1\in S_0\}\\
&=& \inf\{ \frac{c}{c+1},c\geq 0: p-q=c(q_1-p_1), p_1,q_1\in S_0\}\\
&=&[\inf\{ c\geq 0:p-q=c(q_1-p_1)][\inf\{ c\geq 0: p-q=c(q_1-p_1)|}+1]^{-1]\\
&=&\frac{\frac{1}{2}\inf\{2c: p-q=cp_1-cq_1\}}{\frac{1}{2}\inf\{ 2c: p-q=cp_1-cq_1\}+1}\\
&=&\frac{\frac{1}{2}\|p-q\|}{\frac{1}{2}\|p-q\|+1}.
\end{eqnarray*}

From this we get
$ \frac{1}{2}\|p-q\|=\frac{\sigma(p,q)}{1-\sigma(p,q)}=\rho(p,q)$.

Assume $S_0$ is radial, and $p-q=cp_1-dq_1, p,q,p_1,q_1\in S_0$, $c,d\geq 0$. Then
\[ p-q=(c+d)(\frac{c}{c+d}p_1-\frac{d}{c+d}q_1)=(c+d)(p_2-q_2), p_2,q_2\in S_0.
\]
That is, $p-q=b(p_2-q_2), b>0, p_2,q_2\in S_0$. From this we get $\|p-q\|=2\inf\{ c>0, p-q=c(p_1-q_1), p_1,q_1\in S_0\}$ and similarly as in the previous case we obtain
$\|p-q\|=2\rho(p,q)$.

Clearly, if $\|.\|$ is a norm, then $\rho$ is a metric. Suppose $\rho$ is a metric, $\|x-y\|=0, x,y\in X$.

1. $S_0$ is radial.

$\|x-y\|=0$ $\implies$ $\exists p,q\in S_0$, $0\leq c,d\leq 1: x-y=cp-dq=p_1-q_1, p_1,q_1\in S_0$. Then $2\rho(p_1,q_1)=\|p_1-q_1\|=0$ $\implies$ $p_1=q_1\ \implies \ x=y$.

2. $S_0$ is normalized. We show first that $\|p\|=1 \forall p\in S_0$.

If $p=cp_1-dq_1, p_1,q_1\in S_0$, $c,d\geq 0$, then
\[
\frac{1}{1+d}p+\frac{d}{1+d}q_1=c\frac{1}{1+d}p_1\ \implies \ c=1+d \geq 1,
\]
so that $\|p\| =\inf\{ c+d : p=cp_1-dq_1, c,d\geq 0, p_1,q_1\in S_0\}\geq 1$. But also $p=p-0q \ \implies \ \|p\|=1$.

Let $0=\|x-y\|=\|cp-dq\|\geq \|c\|p\|-d\|q\|\|=|c-d|$, so $c=d$. Hence $0=\|x-y\|=c\|p-q\|=2c\rho(p,q)$. If $c\neq 0$ then $\rho(p,q)=0 \ \implies \ p=q$, hence $x=y$.
If $c=0$, then again $x=y$.
\end{proof}

\begin{thebibliography}{WWW}
\bibitem{G} Gudder, S., Convex strustures and operational quantum mechanics, Commun. math. Phys.{\bf 29} (1973) 249--264.
\bibitem{S} M.H. Stone, Postulates for the barycenter calculus, Memiri di M.H. Stone (Chicago, USA)
\bibitem{CF} Capraro, V., Fritz, T., On the axiomatization of convex sybsets of Banach spaces, arXiv:1105.1270v3[math.MG]20Oct.2015
\end{thebibliography}
\end{document} 
