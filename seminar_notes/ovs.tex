
\documentclass[12pt]{article}
\usepackage{amsmath, amsfonts, amsthm, amssymb}
\usepackage[all]{xy}
\newtheorem{lemma}{Lemma}
\newtheorem{thm}{Theorem}
\newtheorem{prop}{Proposition}
\newtheorem{coro}{Corollary}

\theoremstyle{remark}
\newtheorem{rem}{Remark}
\newtheorem{ex}{Example}

\newcommand{\<}{\langle}
\def\>{\rangle}
\newcommand{\ct}[1]{\mathbf{#1}}



\begin{document}

\section{Ordered vector spaces }

Overall reference:  \cite{jameson}

\subsection{Basic definitions}

Let $X$ be a real vector space. A subset $A\subseteq X$ is 
\begin{itemize}
\item algebraically open (closed) if the intersection of any line with $A$ is an open (closed) subset of the line
\item linearly bounded if the intersection of $A$ with any line is  a bounded subset of the line
\end{itemize}
We say that $a\in A$ is an algebraic interior point of $A$ if it is an interior point of the intersection of any line with $A$, that is, 
for any $x\in X$ there is some $\delta>0$ such that $a+sx\in A$ for all $|s|\le \delta$. 
 The set of all such points is called the algebraic interior of $A$ and is denoted by $aint(A)$. The algebraic closure of $A$ is $acl(A):=X\setminus aint(X\setminus A)$.
If $A$ is convex, then
\[
acl(A)=\{x\in X,\ \exists y\in X,\ x+\lambda y\in A,\ \forall \lambda\in (0,1)\}.
\]
$A$ is algebraically open iff $A=aint(A)$ and algebraically closed iff $A=acl(A)$. If $A$ is convex, then both $aint(A)$ and $acl(A)$ are convex as well.

\begin{rem} (cf. \cite[\S 16]{kothe}) 
If $A$ is convex, then $aint(A)$ is algebraically open, but  in general $aint(aint(A))\subsetneq aint(A)$. The algebraic closure is not necessarily algebraically closed even if $A$ is convex. The counterexample is as follows. Let $X$ be an infinite dimensional vector space with algebraic basis $\{x_\alpha\}$. Put 
\[
A=\{ x=\sum_\alpha c_\alpha x_\alpha,\ c_\alpha\ge 0\ \forall \alpha,\ \sum_\alpha c_\alpha \ge \frac1{n(x)}\}
\]
where $n(x)=\#\{\alpha, x=\sum_\alpha c_\alpha x_\alpha,\ c_\alpha\ne 0\}$. Then $A$ is convex, $acl(A)=\{x=\sum_\alpha c_\alpha x_\alpha,\ c_\alpha\ge 0\  \forall \alpha,\ x\ne 0\}$ and $acl(acl(A))$ contains 0, so that $acl(A)\subsetneq acl(acl(A))$.
On the other hand, if $A$ is convex and $aint(A)\ne \emptyset$, then $acl(acl(A))=acl(A)$. \end{rem}

%For $A\subseteq X$, the associated Minkowski functional $\mu_A: X\to \mathbb R^+\cup\{\infty\}$ is defined as 
%\[
%\mu_A(x):=\inf\{\lambda>0, x\in \lambda A\},\qquad x\in X
%\]
%If $A$ is convex and $aint(A)\ne \emtyset$, $\mu_A$ is a seminorm in $X$. 

\subsubsection*{Wedges, cones and orderings}

A subset $P\subseteq X$ is called a  wedge if $P+P\subseteq P$ and $\lambda P\subseteq P$ for any $\lambda\ge 0$. The preorder $x\le y$ if $x-y\in P$ is compatible with the linear structure, such a preorder is called an ordering in $X$. Conversely, for any ordering, the set of positive elements  is a wedge.

The pair $(X,P)$ where $P$ is a wedge is called an ordered vector space. 
The corresponding ordering is a partial order iff $P\cap -P=\{0\}$, in this case $P$ is called a cone. 
$X$ with this ordering is directed  iff $P$ is generating, that is,  $X=P-P$. 

\subsubsection*{Positive maps}

Let $(X,P)$ and $(Y,Q)$ be ordered vector spaces. A  linear map $F:X\to Y$ is called positive if $F(P)\subseteq Q$. If $F$ is invertible with positive inverse, we say that $F$ is an order isomorphism.

 Let  $(P,Q)$ denote the set of positive maps, then $(P,Q)$ is a wedge in the vector space $L(X,Y)$ of all linear maps $X\to Y$. We have

\begin{lemma}\label{lemma:duality_morph} $(P,Q)$ is a cone if and only if $P$ is generating and $Q$ is a cone.

\end{lemma}

\subsubsection*{Archimedean and almost Archimedean orderings}

Let $(X,P)$ be an ordered vector space. We say that the ordering (or $P$) is Archimedean if 
 $x\le \lambda y$ for some $y\in X$ and all $\lambda> 0$ implies that $x\le 0$. 
\begin{prop}
The following are equivalent.
\begin{enumerate}
\item[(i)] the ordering is Archimedean.
\item[(ii)]  $\exists y\in X, \epsilon >0$ such that  $x\le \lambda y$ for all $\epsilon\ge \lambda>0$ $\implies$ $x\le 0$.
\item[(iii)] $\exists y\in P, \epsilon >0$ such that  $x\le \lambda y$ for all $\epsilon\ge \lambda>0$ $\implies$ $x\le 0$.
\item[(iv)] $P=acl(P)$.
\end{enumerate}
\end{prop}

The ordering (or $P$) is almost Archimedean if $-\lambda y\le x\le \lambda y$ for some $y\in X$ and  all $\lambda> 0$ implies  $x=0$.

\begin{prop}
The following are equivalent.
\begin{enumerate}
\item[(i)] the ordering is almost Archimedean.
\item[(ii)]  $\exists y\in X, \epsilon >0$ such that  $-\lambda y\le x\le \lambda y$ for all $\epsilon\ge \lambda>0$ $\implies$ $x= 0$.
\item[(iii)] $\exists y\in P, \epsilon >0$ such that  $-\lambda y\le x\le \lambda y$ for all $\epsilon\ge \lambda>0$ $\implies$ $x= 0$.
\item[(iv)] $acl(P)$ is a cone.
\end{enumerate}
\end{prop}

\begin{rem}
Note that an almost Archimedean wedge must be a cone. An Archimedean  wedge is almost Archimedean iff it is a cone.
\end{rem}






\subsection{Order units and bases}


\subsubsection*{Order units and seminorms}

 An element $u\in X$ is an order unit in $(X,P)$ if for any $x\in X$, there is some $\lambda\in \mathbb R^+$ such that 
 $x\le \lambda u$. This is equivalent to $u\in aint(P)$. If $aint(P)\ne \emptyset$, $P$ is generating.  

If $u$ is an order unit, then $P$ is (almost) Archimedean iff 
$u$ is (almost) Archimedean: $x\le \lambda u$ for all $\lambda>0$ implies $x\le 0$ (resp. $-\lambda u\le x\le \lambda u$ for all $\lambda>0$ implies $x=0$). 



For an order unit $u$, put
 \[
\|x\|_u=\inf\{\lambda>0,\ -\lambda u\le x\le \lambda u\}.
 \]
Then $\|\cdot\|_u$ is a seminorm in $X$. It is a norm iff  $u$ is almost Archimedean. 

\begin{rem}
If $u_1,u_2\in aint(P)$, the associated seminorms $\|\cdot\|_{u_1}$ and $\|\cdot\|_{u_2}$ are equivalent. The corresponding topology is thus a property of the ordering rather than the order unit. In fact, this topology is the finest locally convex topology 
 making all order intervals bounded.
\end{rem}
 

\begin{lemma}\label{lemma:archimedean} Let $u$ be Archimedean. Then $[-u,u]=\{x,\in X,\ \|x\|_u\le 1\}$ and the wedge $P$ is closed in the topology given by $\|\cdot\|_u$.

\end{lemma}

\begin{proof} Let $x\in [-u,u]$, then clearly $\|x\|_u\le 1$. Conversely, assume that $\|x\|_u\le 1$, then $-(1+\epsilon) u\le x\le (1+\epsilon) u$ for all $\epsilon>0$. 
This implies that $\pm x-u\le \epsilon u$ for all $\epsilon>0$ and since $u$ is Archimedean, this implies $\pm x\le u$, that is, $x\in [-u,u]$.

For the second statement, let $x\in \bar P$ (the closure of $P$ w.r. to $\|\cdot\|_u$). Then for all $n\in \mathbb N$, there is some $p_n\in P$ such that $\|x-p_n\|_u\le \tfrac1n$. This implies that $-x\le p_n-x\le \tfrac 1n u$ for all $n$ and since  $u$ is Archimedean, $-x\le 0$, so that $x\in P$.

\end{proof}

Let $(X_i,P_i)$, $i=1,1$ be ordered vector spaces and let $u_i\in X_i$ be order units. A map $f:X_1\to X_2$ is called unital if $f(u_1)=u_2$. The following is immediate.

\begin{prop} Let $(X_i,P_i,u_i)$, $i=1,2$ be order unit spaces. Any positive unital map $f:X_1\to X_2$ is a contraction with respect to the seminorms $\|\cdot\|_{u_1}$ and 
$\|\cdot\|_{u_2}$.
\end{prop}

A triple $(X,P,u)$ where $X$ is a vector space, $P\subseteq X$ an Archimedean  cone and $u\in aint(P)$ is called an order unit space. To summarize, in this case, 
$\|\cdot\|_u$ is a norm in $X$, $[-u,u]$ is the corresponding closed unit ball and $P$ is norm closed. The category of order unit spaces with positive unital maps as morphisms will be denoted by $\ct{AOUS}$.


\subsubsection*{Bases and seminorms}

Let $(X,P)$ be an ordered vector space. A convex subset $K\subset P$ is called a base of $P$ if for any nonzero $p\in P$ there is a unique $\lambda>0$  such that 
$\lambda p\in K$.  

\begin{lemma} Any wedge with a base is a cone.

\end{lemma}

\begin{proof} Let $K$ be a base of a wedge $P$, and let $0\neq x\in P\cap -P$. Then there are $\lambda,\mu >0$ such that $\lambda x=x_1\in K$ and $-\mu x=x_2\in K$.  
It follows that $\lambda^{-1}x_1=-\mu^{-1}x_2$ and then  $\tfrac{\mu}{\lambda+\mu}x_1+\tfrac{\lambda}{\lambda+\mu}x_2=0$. Since $K$ is convex, we obtain $0\in K$, but then 
 for any $p\in K$,  $\lambda p\in K$ for all $\lambda\in[0,1]$. Hence $P$ must be a cone. 

\end{proof}

\begin{prop} \label{prop:base} A wedge  $P$ has a base if and only if there exists a linear functional $\xi$ on $X$ which is strictly positive on $P$. In this case, we may put 
 $K=\{p\in P, \xi(p)=1\}$. 
\end{prop}

\begin{proof}  Let $K$ be a base of $P$. For $p\in P$, let $\xi(p)$ be the unique positive number such that 
$\xi(p)^{-1}p\in K$. Then clearly $\xi(sp)=s\xi(p)$. Further, let $p,q\in P$ and let $\alpha=\xi(p)+\xi(q)$, then 
\[
\alpha^{-1}(p+q)=\frac{\xi(p)}{\alpha} \xi(p)^{-1}p+ \frac{\xi(q)}{\alpha} \xi(q)^{-1}q\in K,
\]
so that  $p\mapsto \xi(p)$ is an additive function  $\xi: P\to \mathbb R^+$. The function $\xi$ easily extends to $P-P$ and has an extension to all of $X$ by Hahn-Banach theorem. This extension is obviously positive and $K=\{p\in P,\ \xi(p)=1\}$.  

Conversely, let $\xi:X\to \mathbb R$ be strictly positive, then $K=\{p\in P, \xi(p)=1\}$ is a convex subset of $P$ and $\xi(p)^{-1}p\in K$ for any $p\in P$. Uniqueness is obvious. 



\end{proof}



\begin{prop} (\cite{ellis}) Let $P$ be a generating cone in a vector space $X$ and let $K$ be a base of $P$. For $x\in X$, put 
\[
\|x\|_K:=\inf\{a+b,\ x=a p-b q,\ a,b\in \mathbb R^+, p,q\in K\}.
\]
This defines a seminorm in $X$, which is a norm if and only if $S:=co(K\cup -K)$ is linearly bounded. 

\end{prop}

\begin{proof}
It can be checked easily that $\|\cdot\|_K$ is a seminorm. Note also that $x\in S$ implies $\|x\|_K\le 1$. Indeed, any $x\in S$ has the form $x=\lambda p-(1-\lambda)q$ for some  $\lambda\in [0,1]$, $p,q\in K$ and then  $\|x\|_K\le \lambda+(1-\lambda)=1$.  
Assume that $\|\cdot\|_K$ is a norm and let $x_t:=x+ty$ be a line in $X$. Then $\|y\|_K>0$ and $x_t\in  S$ implies that $1\ge \|x_t\|_K\ge |\|x\|_K-|t|\|y\|_K|$, so that $|t|\le \tfrac{1+\|x\|_K}{\|y\|_K}$. Conversely, assume that $S$ is linearly bounded and let $\|x\|_K=0$. This implies $tx\in S$ for all $t\in \mathbb R$, hence we must have 
$x=0$. 
 

\end{proof}

The (semi)norm in the above proposition  is called  the base (semi)norm in $X$. 

\begin{rem}\label{rem:minkowski}
Note that $\|\cdot\|_K$ is the Minkowski functional of $S$, that is
\[
\|x\|_K= \inf\{\lambda>0, x\in \lambda S\}.
\]
To see this, observe that $S=\{s p-(1-s)q,\ s\in [0,1],\ p,q\in K\}$. Denote the Minkowski functional by $p_S$. If $x=ap-bq$ for some $a,b\in \mathbb R^+$ and $p,q\in K$, then if $a+b=0$, we must have 
$x=0$ and the equality obviously holds. Otherwise, 
\[
x= (a+b)(\frac a{a+b} p-\frac b{a+b} q)\in (a+b) S,
\]
so that $p_S(x)\le \|x\|_K$. On the other hand, let $x\in \lambda S$ for some $\lambda>0$. Then $x=\lambda(sp-(1-s)q)$ for some $s\in [0,1]$ and $p,q\in K$, so that 
\[
\|x\|_K\le \lambda s+\lambda(1-s)=\lambda,
\]
hence $\|x\|_K\le p_S(x)$.   
\end{rem}




\begin{rem} Linear boundedness of $K$ is in general not enough. There are some weird infinite dimensional examples such that $K$ is linearly bounded but $co(K\cup -K)$ is not.

\end{rem}

Let $(X_i,P_i)$ be ordered vector spaces and let $K_i$ be a base of $P_i$, $i=1,2$.  A linear map $f:X_1\to X_2$ is called base-preserving if $f(K_1)\subseteq K_2$

\begin{prop}  
 Any base-preserving linear  map  is  positive and contractive with respect to the base seminorms.
\end{prop}

A triple $(X,P,K)$, where $X$ is a vector space, $P$ a generating cone and $K$ a base of $P$ such that $co(K\cup-K)$ is linearly bounded is called a base-normed space. 
The category of base-normed spaces with base-preserving linear maps will be denoted by $\ct{BN}$.

\subsubsection*{Some examples}

The wedges $X$ and $\{0\}$ are trivial.

\begin{enumerate}
\item The only nontrivial wedges in $\mathbb R$ are $\mathbb R^+$ and $\mathbb R^{-}$. $\mathbb R$ with the usual ordering and norm 
 concides with both the order unit space $(\mathbb R,\mathbb R^+,1)$ and the base-normed space $(\mathbb R,\mathbb R^+,\{1\})$.
\item \label{ex:fs}\textbf{Function spaces:} Let $S$ be a set, $X=\{f:S\to \mathbb R\}$, $P=\{f, f(S)\subseteq \mathbb R^+\}$. $P$ is an Archimedean cone, $(X,\le)$ is a lattice. If $S$ is not finite, $aint(P)=\emptyset$.
\item  \label{ex:fsb} As \ref{ex:fs}, but bounded functions.  In this case $P$ is an Archimedean cone, $aint(P)$ is the set of strictly positive functions.
\item \label{ex:affine} Let  $K$ be a convex set, we will denote the set of all affine functions $K\to \mathbb R$ by $A(K)$, 
the set of bounded affine functions by $A_b(K)$.  If $K$ is also a topological space, we denote the set of continuous affine functions by $A_c(K)$
 We also denote by $A(K)^+$ ($A_b(K)^+$, $A_c(K)^+$) the set of positive affine (bounded, continuous) functions and $1_K$ the constant function $1_K(x)\equiv 1$. Then $(A_{b}(K),A_{b}^+(K),1_K)$ is an order unit space, with $\|f\|_{1_K}=\sup_{x\in K}|f(x)|$. If 
  $K$ is also a compact Hausdorff topological space, then the same is true for $(A_{c}(K),A_{c}^+(K),1_K)$.
 \item $X=\{f:\mathbb R\to \mathbb R\} $, with the cone of nondecreasing functions.
\item \textbf{Sequence spaces:} $X$ the set of all  (or bounded, summable, convergent, converging to 0,...) sequences, with usual positive cone.
\item $\mathbb R^2$ with the usual or lexicographic ordering, with $P=\{(x,y),\ x>0, y>0\}\cup \{0\}$ or $P=\{(x,y),\ x>0\}\cup \{0\}$.
\end{enumerate}



\subsubsection*{Completeness}

We give some sufficient conditions for completeness of order unit norms and base norms.

\begin{prop}\cite{jameson} Let $(X,P)$ be an ordered vector space with an almost Archimedean order unit $u$. If every majorized increasing sequence in $(X,P)$ has a supremum, then $(X,\|\cdot\|_u)$ is complete.


\end{prop}


\begin{proof} We first show that any increasing Cauchy sequence has a limit. So let $\{x_n\}$ be such a sequence and let $\epsilon>0$. Then $\|x_n-x_m\|_u< \epsilon$ for $m,n\ge N$. We then have for all $m\ge N$, $x_m-x_N\le \epsilon u$, so that $x_m\le x_N+\epsilon u$. It follows that $\{x_n\}$ is a majorized increasing sequence, so that 
there is some $x_0$ such that $x_0=\sup_n x_n$. For all $m,n \ge N$, we have $x_n\le x_m+\epsilon u$, hence $x_0\le x_m+\epsilon u$ and we have 
$0\le x_0-x_m\le \epsilon u$. This implies $\|x_0-x_m\|_u\le \epsilon$ for all $m\ge N$, so that $\lim_n x_n=x_0$. 

Let now $\{x_n\}$ be any Cauchy sequence. Let $V_n=\{p-q,\ p,q\in [0,2^{-n}]\}$, then $V_n$ contains the ball with center 0 and radius $2^{-{n+1}}$ and is therefore a neighborhood of 0.  Hence there is a subsequence such that $x_n-x_{n-1}\in V_n$. Let $a_n,b_n\in [0,2^{-n}]$ be such that $x_n-x_{n-1}=a_n-b_n$. Then $\{\sum_{k=1}^n a_k\}$
 and $\{\sum_{k=1}^n b_k\}$ are increasing Cauchy sequences and hence have a limit by the first part of the proof. Moreover, we have 
$x_n=\sum_{k=1}^n(a_k-b_k)$, so that $x_n$ converges as well.



\end{proof}



\begin{prop}\label{prop:bn_complete}\cite{alfsen}  Let $K$ be a base of $P$  and asume that $K$ is compact with respect to some Hausdorff topology $\tau$, compatible with the linear structure of $X$. Then $X$ is $\|\cdot\|_K$-complete.

\end{prop}

\begin{proof} Note that $S=co(K\cup-K)$ is also $\tau$-compact, hence  must be linearly bounded. It follows that $\|\cdot\|_K$ is a norm and it is easy to verify that $S$ is the closed unit ball. Let $\{x_n\}$ be a Cauchy sequence, then it is norm-bounded, so we may assume that $\{x_n\}\subset S$.  Let $y\in S$ be a $\tau$-accumulation point of $\{x_n\}$. For $\epsilon>0$, $\|x_n-x_m\|_K<\epsilon$ for $n,m\ge N$. This implies that
$x_n\in x_N+\epsilon S$ for $n\ge N$. Since $S$ is $\tau$-closed,  $y\in x_N+\epsilon S$. It follows  that
\[
\|y-x_n\|_K\le \|y-x_N\|_K+\|x_N-x_n\|_k\le 2\epsilon,
\]
this finishes the proof.


\end{proof}






\subsection{Duality}
\subsubsection*{Positive functionals}

Let $(X,P)$ be an ordered vector space and let $X'$ denote the algebraic dual of $X$. The dual wedge of $P$ is defined as
\[
P':=\{\varphi\in X', \varphi(p)\ge 0, \forall p\in P\}
\]
  Note that  $P'=(P,\mathbb R^+)$ and it follows by  Lemma \ref{lemma:duality_morph} that 
	$P'$ is a cone iff $P$ is generating. 
	
\begin{rem}
To see the above duality in this specific case, note that $P$ is a generating wedge in the subspace $P-P$, whose algebraic dual can be identified with the quotient space $X'|_{(P-P)^\perp}$. Here
\[
(P-P)^\perp=\{\varphi\in X',\ \varphi(x)=0,\ \forall x\in P-P\}=P'\cap-P'.
\]
If $P-P=X$, then $P'\cap-P'=(P-P)^\perp=\{0\}$, so $P'$ is a cone. Conversely, if $P'$ is a cone, then 
\[
P-P=(P-P)^{\perp\perp}=\{0\}^\perp=X,
\]
this holds since any subspace $E\subseteq X$ satisfies $E^{\perp\perp}=E$. However, this is no longer true for subspaces in $X'$ (\cite{kothe}), so a dual statement does not hold. More precisely, it is easily checked that $P\cap -P\subseteq (P'-P')^\perp$, so that 
 if $P'$ is generating, $P$ must be a cone. The converse is not true in general: we only have $P'-P'\subseteq (P'-P')^{\perp\perp}=
 (P\cap -P)^\perp$, so $P'$ may be not generating even if $P$ is a cone (there are indeed counterexamples).
\end{rem}

It is also clear from the definition that $P'$ is  Archimedean.


\subsubsection*{The  dual of a vector space with an order unit } 


Let $(X,P)$ be an ordered vector space with an order unit $u$. Positive unital linear functionals are called states, the set of all 
 states will be denoted by $\mathcal S(X,P,u)$. We will study the subspace of $\|\cdot\|_u$-bounded functionals
\[
X^*:=\{\varphi\in X', \ \|\varphi\|_u^*:=\sup_{\|x\|_u\le 1}|\varphi(x)|<\infty\}.
\]
Note that $\|\cdot\|_u^*$ is a norm in $X^*$ and $(X,\|\cdot\|_u^*)$ is a Banach space.

\begin{lemma}\label{lemma:states}
\begin{enumerate}
\item[(i)] Any $\varphi\in P'$ is $\|\cdot\|_u$-bounded, with 
\[
\|\varphi\|_u^*:=\sup_{\|x\|_u\le 1}|\varphi(x)|=\varphi(u)
\]
\item[(ii)] $\mathcal S(X,P,u)$ is a base of $P'$.
\item[(iii)] If $\varphi\in X'$ is such that $\|\varphi\|_u^*=\varphi(u)$, then $\varphi\in P'$.
\item[(iv)] For $x\in X$, $\|x\|_u=\sup_{\varphi\in \mathcal S(X,P,u)} |\varphi(x)|$.
\end{enumerate}


\end{lemma}
\begin{proof}
 (i) is quite easy. This also implies that $u$ is strictly positive over $P'$, hence the set of states forms a base of $P'$ by Proposition \ref{prop:base}. For (iii), we may assume $\varphi(u)=1$. Let $x\in P$ and let $\lambda>0$ be such that $0\le x\le \lambda u$. Then
 $\|x-\lambda u\|_u\le \lambda$ and we have
 \[
|\varphi(x)-\lambda|=|\varphi(x-\lambda u)|\le \|\varphi\|_u^*\|x-\lambda u\|_u\le \lambda.
 \]
This implies $\varphi(x)\ge 0$. For (iv), let $x\in X$ be such that  $-\lambda u\le x\le \lambda u$, then $|\varphi(x)|\le \lambda$ for any $\varphi\in \mathcal S(X,P,u)$, so that $\sup_{\varphi\in \mathcal S(X,P,u)} |\varphi(x)|\le \|x\|_u$. Assume that this inequality is strict for some $x_0$, we may put $\|x_0\|_u=1$. Then there is some $0<a<1$ such that $|\varphi(x_0)|< a$ for any state $\varphi$. Let $H=\{x\in X, x\le au\}$, then we have either $x_0\notin H$ or $-x_0\notin H$. Note that $H=au-P$ is a convex set such that $aint(H)\ne \emptyset$. If $y\notin H$, then by a separation theorem by Edelheit \cite[0.2.4]{jameson}, there is some nonzero $\psi\in X'$ such that $\sup_{x\in H}\psi(x)\le \psi(y)$. This implies that $\psi$ is bounded below on $P$, so that we must have $\psi\in P'$. 
 Normalizing, we may assume that $\psi\in \mathcal S(X,P,u)$. Then
 \[
a=\psi(au)\le\sup_{x\in H}\psi(x)\le \psi(y).
 \]
Since we may take either $x_0$ or $-x_0$ for $y$, we have arrived at a contradiction.

\end{proof}



\begin{thm}\label{thm:ou_dual} Let $(X,P)$ be an ordered vector space with an order unit $u$ and let $K=\mathcal S(X,P,u)$.
Then $X^*=P'-P'$   and  $(X^*,P',K)$  is a base-normed space, with $\|\cdot\|_K=\|\cdot\|_u^*$. 

\end{thm}


\begin{proof}\cite{ellis} By Lemma \ref{lemma:states}  any $\varphi\in P'-P'$ is $\|\cdot\|_u$-bounded. 
For the converse, let $Y=X\times X$ be ordered by the wedge $Q=P\times P$, then $(u,u)$ is an order unit in $(Y,Q)$.
Let 
\[
Z=\{t(u,u)-(x,-x),\ t\in \mathbb R, x\in X\},
\]
then $Z$ is a linear subspace in $Y$ containing the order unit. 
Let  $\varphi\in X'$ be $\|\cdot\|_u$-bounded and  put 
\[
F_\varphi(z)=t\|\varphi\|_u^*-\varphi(x), \qquad z=t(u,u)-(x,-x)\in Z
\]
This defines a linear functional on $Z$. Moreover, note that $z=t(u,u)-(x,-x)\in Q$ iff $\|x\|_u\le t$ and then $F_\varphi(z)\ge (t-\|x\|_u)\|\varphi\|_u^*\ge 0$.
Since $Z$ contains the order unit, $F_\varphi$ extends to a positive linear functional on $Y$ (\cite[Corollary 1.6.2]{jameson}). Put 
\[
\psi_1(x)=F_\varphi(x,0),\quad \psi_2(x)=F_\varphi(0,x),\qquad x\in X.
\]
Then $\psi_1,\psi_2\in P'$ and $\varphi=\psi_2-\psi_1$, so that $\varphi\in P'-P'$. We have 
\[
\|\varphi\|_u^*=F_\varphi(u,u)= \psi_1(u)+\psi_2(u)\ge \|\varphi\|_K
\]
On the other hand, let $\varphi=a\varphi_1- b\varphi_2$ with $a,b\ge 0$, $\varphi_1,\varphi_2\in K$, then
$\|\varphi\|^*_u\le a+b$, this shows that $\|\varphi\|_u^*=\|\varphi\|_K$. To finish the proof, we have to show that $\|\cdot\|_K$ is a norm. So let $\|\varphi\|_K=0$, then $\varphi$ is zero over the absorbing set $[-u,u]$, so that $\varphi=0$.



\end{proof}

\begin{coro} The norm dual of an ordered vector space with order unit norm is a base-normed space, with base formed by the set of states.

\end{coro}

\begin{coro} Let $(X,P)$ be an ordered vector space with an order unit $u$. 
\begin{enumerate}
\item[(i)] $P$ is almost Archimedean iff the set of states is separating.
\item[(ii)] $P$ is  Archimedean iff the set of states is order-determining.
\end{enumerate}


\end{coro}


\begin{proof} (i) is immediate from Lemma \ref{lemma:states} (i). For (ii), assume that $P$ is an Archimedean wedge and let $x\in X$ be such that $\varphi(x)\ge 0$ for all
 states $\varphi$.   Assume $x\notin P$, then, since $P$ is algebraically closed and $aint(P)\neq \emptyset$, there exists a functional $\psi\in X'$ such that 
$\psi(x)<c\le \inf_{p\in P} \psi(p)$ (\cite[17.5.(2)]{kothe}). This implies that $\psi(p)\ge 0$ for all $p\in P$, so that $\psi\in P'$. By normalization,  we obtain a state with $\psi(x)<0$, a contradiction.

 Conversely, let $L=\{x_t\}$ be any line 
in $X$, then $x_t\in P$ iff $\varphi(x_t)\ge 0$ for all states $\varphi$, so that $L\cap P$ is a closed subset in $L$, hence $P$ is algebraically closed.


\end{proof}






\subsubsection*{The dual of an ordered vector space with a based cone}


Let $(X,P)$ be an ordered vector space and let $K\subset P$ be a base of $P$. By Proposition \ref{prop:base}, there is a strictly positive linear functional $u\in X'$, such that $K=\{p\in P, u(p)=1\}$. Again, we study the subspace of $\|\cdot\|_K$-bounded functionals
\[
X^*=\{\varphi\in X',\ \|\varphi\|_K^*:=\sup_{x\in P-P,\ \|x\|_K\le 1} |\varphi(x)|<\infty\}.
\] 
Note that here the seminorm is defined only on the subspace $P-P$, so that $\|\cdot\|_K^*$ is in general only a seminorm. 


\begin{thm} \label{thm:base_dual} 
\begin{enumerate}
 \item[(i)] Let $\varphi\in X'$, then $\varphi\in X^*$ iff $\varphi$ is bounded over $K$ and 
\[
\|\varphi\|_K^*=\sup_{x\in K}|\varphi(x)|.
\]
\item[(ii)] Let $P^*=P'\cap X^*$, then  $u$ is an Archimedean order unit in $(X^*,P^*)$ and $\|\cdot\|_u=\|\cdot\|_K^*$.
\item[(iii)]
If $X=P-P$, then  
$(X^*,P^*,u)$ is  an order unit space,  isomorphic to $(A_b(K),A_b(K)^+,1_K)$ (see Example \ref{ex:affine}) (we mean a unital order isomorphism).
\end{enumerate}

\end{thm}

\begin{proof} Let $\varphi\in X'$ and let $x=ax_1-bx_2$ with $a,b\ge 0$ and $x_1,x_2\in K$. Then
\[
|\varphi(x)|\le a|\varphi(x_1)|+b|\varphi(x_2)|\le (a+b)\sup_{x\in K}|\varphi(x)|.
\]
It follows that $\varphi\in X^*$ if it is bounded over $K$, and also that $\|\varphi\|_K^*\le\sup_{x\in K}|\varphi(x)|$. The rest of (i) is easy.

 Since the dual wedge is Archimedean, $P^*$ is Archimedean as well and $u\in P^*$ by (i). 
We also obtain that $\|\varphi\|_K^*\le M$ iff $-Mu(x)\le  \varphi(x)\le Mu(x)$ for all $x\in K$  and since $K$ is a base of $P$, this is equivalent to  
 $-Mu\le \varphi\le Mu$, so that $u$ is an order unit in $(X^*,P^*)$, with $\|\varphi\|_u= \|\varphi\|_K^*$, this proves (ii).

For (iii), assume that $X=P-P$. Then $P'$, and hence also $P^*$, is an Archimedean cone, so that $(X^*,P^*,u)$ is an order unit space.
Further, any element $\varphi\in X'$ restricts to a function $\varphi|_K\in A(K)$ and conversely, any 
function in $A(K)$ extends to a unique linear functional in $X'$. It is easy to see  from (i) and (ii) that this defines a unital order isomorphism between $(X^*,P^*,u)$ and 
$(A_b(K),A_b(K)^+,1_K)$.

\end{proof}

\begin{coro} The norm dual of a base-normed space is an order unit space.

\end{coro}




\subsubsection*{Preduals}

We next discuss the Banach space preduals of order unit and base-normed spaces. Below,  $(X,\|\cdot\|)$ is a Banach space and
$(X^*,\|\cdot\|^*)$ its norm dual.  If $P\in X$ is a wedge, we will denote 
\[
P^*:=\{\varphi\in X^*,\ \varphi(p)\ge 0,\ \forall p\in P\}= P'\cap X^*.
\]
Similarly, if $Q$ is a wedge in $X^*$, we will denote 
\[
Q_*:=\{x\in X,\ q(x)\ge 0,\ \forall q\in Q\}= Q'\cap X.
\]
It is clear that $P^*$ and $Q_*$ are wedges. Moreover, $(P^*)_*=\bar P$ is the norm=weak closure of $P$  and $(Q_*)^*$ is the weak*-closure of $Q$. 

\begin{thm}\cite{ellis, asell} If $X^*$ is an order unit space, then $X$ is base-normed. More precisely, 
if there is an Archimedean cone $Q\subset X^*$ with an order unit $u$  such that $\|\cdot\|^*=\|\cdot\|_u$, 
 then  $Q_*\subset X$ has a base $K=\{p\in Q_*, u(p)=1\}$ and $(X,Q_*,K)$ is a base-normed 
space with $\|\cdot\|=\|\cdot\|_K$.

\end{thm}

\begin{proof}
Note first that the cone $Q^*$ is weak*-closed. Indeed, by Lemma \ref{lemma:archimedean}, $[-u,u]$ is the closed unit ball $B^*_1$ in $(X^*,\|\cdot\|^*)$ and is therefore weak*-compact by Banach-Alaoglu theorem. Hence also $u+[-u,u]=[0,2u]$ is weak*-compact, so that 
this is true for $Q\cap \rho B_1^*= [0,\rho u]$ for any $\rho\ge 0$. By Krein-Smulyan theorem, this implies that $Q$ is weak*-closed.

Let $p\in Q_*$ be such that $u(p)=0$, then for any $\varphi\in Q$, 
\[
0\le \varphi(p)\le \|\varphi\|_uu(p)=0.
\]
Since  $X^*=Q-Q$ separates points in $X$, we obtain $p=0$. Hence 
$u$ defines a strictly positive linear functional on $(X,Q_*)$  and $K$ is a base of $Q_*$. 
For $p\in Q_*$, we have
\[
\|p\|=\sup_{\varphi\in [-u,u]}|\varphi(p)|=u(p).
\]
It follows that $S=co(K\cup -K)$ is a subset of the closed unit ball $B_1$ of $X$, so that $\|\cdot\|\le \|\cdot\|_K$ (since $\|\cdot\|_K$ is the Minkowski functional of $S$ and 
$\|\cdot\|$ is the Minkowski functional of $B_1$).   We next show that $S$ is dense in $B_1$. Since $Q=(Q_*)^*$, we have for $\varphi\in X^*$:
\begin{align*}
\|\varphi\|_u&=\inf\{\lambda >0,\ \lambda u\pm \varphi\in Q\}=\inf\{\lambda >0,\ (\lambda u\pm \varphi)(p)\ge 0,\ \forall p \in Q_*\}\\
&= \inf\{\lambda>0,\ |\varphi(p)|\le \lambda,\ \forall p\in K\}=\sup_{p\in K}|\varphi(p)|.
\end{align*}
Assume that $x_0\in B_1\setminus \bar S$, then by Hahn-Banach separation theorem, there is some $\varphi\in X^*$ such that 
\[
\|\varphi\|_u=\sup_{p\in K}|\varphi(p)|\le \sup_{x\in S}\varphi(x)< \varphi(x_0)\le \|\varphi\|^*=\|\varphi\|_u,
\]
a contradiction. It follows that $\bar S=B_1$.  

Choose any $\alpha>1$ and let $\alpha_n>0$ be a sequence such that  $1+\sum_n\alpha_n<\alpha$. For $x_0\in B_1=\bar S$ there is some element $x_1\in S$ such that $\|x_0-x_1\|< \alpha_1$, that is, $x_0-x_1\in \alpha_1B_1$. But then similarly there is some $x_2\in \alpha_1S$ 
such that $\|x_0-x_1-x_2\|<\alpha_2$. Continuing by induction, we obtain a sequence $\{x_n\}$ in $X$ such that $\|x_n\|_K\le \alpha_{n-1}$ and  $\|x_0-\sum_n x_n\|<\alpha_n\to 0$. Hence  
\[
\|x_0\|_K=\|\sum_n x_n\|_K\le \sum_n\|x_n\|_K\le 1+\sum_n\alpha_n<\alpha,
\]
so that  $x_0\in \alpha S$. Since the above inequality holds for all $\alpha>1$, we have
$1=\|x_0\|\le \|x_0\|_K\le 1$. It also  follows that  $B_1\subset \alpha S$ for any $\alpha>1$ and consequently $X=Q_*-Q_*$. 





\end{proof}


\begin{thm} Let $(X^*,\|\cdot\|^*)$ be a base-normed space with a positive cone $Q$ having a weak*-compact base $K$. Let $u$ be a strictly positive functional corresponding to $K$.  
Then $u\in X$ and $(X,Q_*,u)$ is an order unit space isomorphic to $(A_c(K),A_c(K)^+,1_K)$, with  $\|\cdot\|_u=\|\cdot\|$.

\end{thm}

\begin{proof}  By Theorem \ref{thm:base_dual}, $(X^{**},Q^*,u)$ is an order unit space  isomorphic to $(A_b(K),A_b(K)^+,1_K)$
 and $\|\cdot\|^{**}=\|\cdot\|_u$. Since $X$ can be identified with the subspace of weak*-continuous functionals in $X^{**}$, it is enough to prove that $\phi\in X^{**}$ is weak*-continuous iff $\phi|_K\in A_c(K)$. 

So assume the latter. It is enough to show that $\phi^{-1}(0)$ is weak*-closed in $X^*$. By Krein-Smulian theorem, this is equivalent to 
\[
A=\{\varphi\in X^*,\ \phi(\varphi)=0,\ \|\varphi\|_K\le 1\}
\]
being weak*-closed. Clearly, $\varphi\in A$ iff $\varphi=a(\psi_1-\psi_2)$, with $\psi_1,\psi_2\in K$ and $0\le a\le 1/2$. Let $\psi\in K$ be any element, then $\varphi= 1/2(2a\psi_1+(1-2a)\psi -(2a\psi_2+(1-2a)\psi)$, it follows that $A=1/2(K-K)$ 
 is weak*-closed and hence $\phi\in X$. The converse is obvious.


\end{proof}

\begin{coro} Any order unit space $(X,P,u)$ is isomorphis to a subspace of $(A_c(K), A_c(K)^+,1_K)$, where $K=\mathcal S(X,P,u)$.

\end{coro}


%\subsection{Categories of ordered vector spaces}

\begin{thebibliography}{99}
\bibitem{alfsen} E. M. Alfsen, \emph{Compact convex sets and boundary integrals}, Springer, 1971
\bibitem{asell} L. Asimow and A. J. Ellis, \emph{ Convexity Theory and its Applications in Functional Analysis}, Academic Press, London, 1980
\bibitem{ellis} A. J. Ellis, The duality of partially ordered normed linear spaces, J. London Math. Soc. \textbf{39} (1964), 730-744
\bibitem{jameson} 
G. Jameson, \emph{Ordered Linear Spaces}, Lecture Notes in Mathematics, Springer, 1970
\bibitem{kothe} G. K\" othe, \emph{Topological Vector Spaces}, Springer, 1983
\end{thebibliography}


\end{document}

