\documentclass{amsart}
\usepackage[utf8]{inputenc}
\usepackage[T1]{fontenc}
\usepackage{ifpdf}
\ifpdf
\usepackage[all,pdf,2cell]{xy}\UseAllTwocells\SilentMatrices
\else
\usepackage[all,xdvi,2cell]{xy}\UseAllTwocells\SilentMatrices
\fi
\newcommand{\EA}{\mathbf{EA}}
\newcommand{\Conv}{\mathbf{Conv}}
\newcommand{\Convop}{\mathbf{Conv}^{op}}
\newcommand{\states}{\Sigma}
\newcommand{\effects}{E}

\begin{document}
\title{The state-effect monad on effect algebras}

\section{Preliminaries}
\subsection{Notation}
\begin{itemize}
\item $\EA$ is category of effect algebras.
\item $\Conv$ is the category of convex sets (whatever that is).
\item $\states:\EA\to\Convop$ is the state space functor.
\item $\effects:\Convop\to\EA$ is the effects functor.
\end{itemize}
\subsection{The adjunction}
Let $A$ be an effect algebra. We write $\states(A)$ for the convex set consisting of all states
on $A$, equipped with convex combinations defined pointwise. If $f:A\to B$ is a morphism
of effect algebras, then $\states(f):\states(B)\to\states(A)$ is given by the 
rule $\states(f)(s)=s\circ f$. The mapping $\states(f)(s)$ is clearly a
state on $A$ and, whenever $s,s'$ are two states on $B$, $\theta\in[0,1]$ and $b\in B$,
\begin{align*}
(\states(f)(\theta s+(1-\theta)s'))(b)=\\
((\theta(s)+(1-\theta)s')\circ f)(b)=
((\theta(s)+(1-\theta)s')(f(b)))=\\
\theta(s(f(b)))+(1-\theta)(s'(f(b)))=
\theta((s\circ f)(b))+(1-\theta)((s' \circ f)(b))=\\
\theta((\states(f)(s))(b))+(1-\theta)((\states(f)(s'))(b))
\end{align*}
hence
$$
\states(f)(\theta s+(1-\theta)s')=
\theta(\states(f)(s))+(1-\theta)(\states(f)(s'))
$$
so $\states(f)$ is a morphism in $\Convop$.

Let $K$ be a convex set. An effect $\phi:K\to[0,1]$ on $K$ is an affine mapping 
(that means, a morphism in $\Conv$) into a line segment $[0,1]$. Clearly, the set of all effects on $K$ 
(denoted by $\effects(K)$) can be equipped with a pointwise partial addition inherited from the effect algebra 
$[0,1]$. If $f:K\to K'$ is a morphism in $\Conv$, then $E(f):E(K')\to E(K)$ is given by the rule
$E(f)(\phi)=\phi\circ f$. Let us prove that this is indeed a morphism of effect algebras
\end{document}
