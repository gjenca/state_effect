\documentclass{amsart}
\usepackage[utf8]{inputenc}
\usepackage[T1]{fontenc}
\usepackage{ifpdf}
\ifpdf
\usepackage[all,pdf,2cell]{xy}\UseAllTwocells\SilentMatrices
\else
\usepackage[all,xdvi,2cell]{xy}\UseAllTwocells\SilentMatrices
\fi
\newcommand{\EA}{\mathbf{EA}}
\newcommand{\Conv}{\mathbf{Conv}}
\newcommand{\Convop}{\mathbf{Conv}^{op}}
\newcommand{\states}{\Sigma}
\newcommand{\effects}{E}

\begin{document}
\title{The state-effect monad on effect algebras}

\section{Preliminaries}
\subsection{Notation}
\begin{itemize}
\item $\EA$ is category of effect algebras.
\item $\Conv$ is the category of convex sets (whatever that is).
\item $\states:\EA\to\Convop$ is the state space functor.
\item $\effects:\Convop\to\EA$ is the effects functor.
\end{itemize}
\subsection{The adjunction}
Let $A$ be an effect algebra. We write $\states(A)$ for the convex set consisting of all states
on $A$, equipped with convex combinations defined pointwise. If $f:A\to B$ is a morphism
of effect algebras, then $\states(f):\states(B)\to\states(A)$ is given by the 
rule $\states(f)(s)=s\circ f$. The mapping $\states(f)(s)$ is clearly a
state on $A$ and, whenever $s,s'$ are two states on $B$, $\theta\in[0,1]$ and $a\in A$,
\begin{align*}
\bigl(\states(f)\bigl(\theta s+(1-\theta)s'\bigr)\bigr)(a)=\\
\bigl(\bigl(\theta(s)+(1-\theta)s'\bigr)\circ f\bigr)(a)=
\bigl(\bigl(\theta(s)+(1-\theta)s'\bigr)\bigl(f(a)\bigr)\bigr)=\\
\theta\bigl(s\bigl(f(a)\bigr)\bigr)+(1-\theta)\bigl(s'\bigl(f(a)\bigr)\bigr)=
\theta\bigl((s\circ f)(a)\bigr)+(1-\theta)\bigl((s' \circ f)(a)\bigr)=\\
\theta\bigl(\bigl(\states(f)(s)\bigr)(a)\bigr)+(1-\theta)\bigl(\bigl(\states(f)(s')\bigr)(a)\bigr)
\end{align*}
meaning that
$$
\states(f)(\theta s+(1-\theta)s')=
\theta(\states(f)(s))+(1-\theta)(\states(f)(s'))
$$
so $\states(f)$ is a morphism in $\Convop$.

Let $K$ be a convex set. An effect $\phi:K\to[0,1]$ on $K$ is an affine mapping 
(that means, a morphism in $\Conv$) into a line segment $[0,1]$. Clearly, the set of all effects on $K$ 
(denoted by $\effects(K)$) can be equipped with a pointwise partial addition inherited from the effect algebra 
$[0,1]$. In detail, if $\phi,\psi\in\effects(K)$, then $\phi\oplus\psi$ exists in $\effects(K)$ if and
only if, for all points $x\in X$, $\phi(x)+\psi(x)\leq 1$ and then $(\phi\oplus\psi)(x)=\phi(x)\oplus\psi(x)$.

If $f:K\to K'$ is a morphism in $\Conv$, then $\effects(f):\effects(K')\to \effects(K)$ is given by the rule
$\effects(f)(\phi)=\phi\circ f$. Let us prove that this is indeed a morphism of effect algebras.
Suppose that $\phi\oplus\psi$ exists in $\effects(K')$. Then, for all $x\in K$,
$$
(\effects(f)(\phi\oplus\psi))(x)=((\phi\oplus\psi)\circ f)(x)=
(\phi\oplus\psi)(f(x))=\phi(f(x))+\psi(f(x))\leq 1,
$$
because $\phi\oplus\psi$ exists in $\effects(K')$. Thus,
$$
\phi(f(x))+\psi(f(x))=(\phi\circ f)(x)+(\psi\circ f)(x)=
((\effects(f))(\phi))(x)+((\effects)(f))(\psi)(x)\leq 1
$$
Therefore, $\effects(f)(\phi)\oplus\effects(f)(\psi)$ exists
in $\effects(K)$ and $\effects(f)(\phi)\oplus\effects(f)(\psi)=
\effects(f)(\phi\oplus\psi)$. Moreover, it is easy to see that $\effects(f)(1)=1$,
so $\effects(f)$ is a morphism of effect algebras.
\end{document}
