\documentclass[12pt]{article}
\usepackage{amsmath, amsfonts, amsthm}
\usepackage[all]{xy}
\newtheorem{lemma}{Lemma}
\newtheorem{thm}{Theorem}
\newtheorem{prop}{Proposition}
\newcommand{\<}{\langle}
\def\>{\rangle}
\newcommand{\ct}[1]{\mathbf{#1}}


\begin{document}

\section{Categories and functors}

Here we list some categories and relations between them. 

\subsection{Ordered vector spaces}

The category of ordered vector spaces with an order unit, with positive unital maps as morphisms will be denoted by $\ct{OVSu}$ and the full subcategory of order unit 
spaces by  $\ct{OUS}$. An order unit space which is complete with respect to the order unit norm will be called an order unit Banach space. The full subcategory of these spaces will be denoted by $\ct{BOUS}$. $\ct{BOUS}$ is also a (non-full) subcategory of $\ct{Ban}$, the category of Banach spaces with contractios.


Let $\ct{BN}$ denote the category of base-normed spaces with positive base-preserving maps and $\ct{BBN}$ the subcategory of complete base-normed spaces. 
 This is another subcategory of $\ct{Ban}$. Taking the adjoints defines  (contravariant) functors between  $\ct{BBN}$ and $\ct{BOUS}$. 

\subsection{Convex structures}

A general convex set, or a convex structure, was defined in \cite{gudder} as a set $X$ endowed with a ternary operation $\<\cdot,\cdot,\cdot\>: [0,1]\times X\times X\to X$,
 satisfying certain relations. For subsets in a vector space, we may put $\<s,x,y\>=sx+(1-s)y$ for $s\in [0,1]$ and $x,y\in X$, to obtain the usual definition of a convex set.  If $(X,\<\cdot,\cdot,\cdot\>_X)$ and $(Y,\<\cdot,\cdot,\cdot\>_Y)$ are convex structures, a map $f:X\to Y$ is called affine if it satisfies
\[
\<s,f(x),f(y)\>_Y=f(\<s,x,y\>_X),\qquad s\in [0,1],\ x,y\in X.
\]
The category of convex structures with affine maps as morphisms will be denoted by $\ct{Conv}$, the full subcategory of bounded convex sets is denoted by $\ct{BConv}$. The objects of $\ct{CConv}$ are convex subsets of a vector space,  compact with respect to some locally convex topology, with affine continuous maps as morphisms.

\begin{prop} The categories $\ct{Bconv}$ and $\ct{BN}$ are equivalent.

\end{prop}

%netreba asi
\begin{prop}\cite{} The categories $\ct{CConv}$ and $\ct{BOUS}$ are dually equivalent.

\end{prop}









\subsection{Effect algebras and effect modules}

Let $\ct{EA}$ denote the category of effect algebras and $\ct{EMod}$ the category of effect modules (convex effect algebras). By the results of \cite{sylvia_plus3}, 
$\ct{EMod}$ is a full subcategory of $\ct{EA}$. We also have

\begin{prop}\label{prop:EMod_OVSu}   $\ct{EMod}$ and $\ct{OVSu}$ are equivalent categories.

\end{prop}

\begin{proof} \cite{sylvia_plus3} 

\end{proof}

Let $V: \ct{EMod}\to \ct{OVSu}$ be the functor defined in \cite{sylvia_plus3}. We say that $A\in \ct{EMod}$ is Archimedean if $VA$ is an order unit space and $A$ is called 
a Banach effect module  if $VA$ is an  order unit Banach space. Banach effect modules are a full subcategory of $\ct{EMod}$ (and of course also of $\ct{EA}$), denoted by 
$\ct{BEMod}$.

\begin{prop} Let $A\in \ct{EMod}$. Then 
\begin{enumerate}
\item[(i)] $A$ is Archimedean iff the set of states $\Sigma(A)$ is order-determining.
\item[(ii)]\cite{roroy} $A$ is a Banach effect module iff, in addition to (i), $A$ is complete in the metric
\[
d(a,b)=\sup_{s\in \Sigma(A)} |s(a)-s(b)|.
\] 
\end{enumerate}


\end{prop}

\begin{prop} $\ct{BEMod}$ is a reflective subcategory in $\ct{EA}$.


\end{prop}

\begin{proof}

\end{proof}


\subsection{$\sigma$-effect modules}

A monotone $\sigma$-complete effect module is called a $\sigma$-effect module. If $A$ and $B$ are $\sigma$-effect modules, then 
 any morphism $f:A\to B$ is a $\sigma$-morphism if $f(\vee a_i)=\vee f(a_i)$ for every increasing sequence $\{a_i\}\in A$. A state
 $s\in \Sigma(A)$ which is a $\sigma$-morphism $A\to [0,1]$ is called a $\sigma$-state.





\section{Convex effect algebras and ordered vector spaces}

Let $V: CEA\to OVSu$ be the functor defined in \cite{sylviaetc}. Here $OVSu$ is the category of ordered vector spaces with an order unit and positive unital linear maps. 
We will now introduce a monad on $EA$ such that the corresponding algebras are precisely the complete archimedean effect algebras. Let $CConv$ be the category whose objects are compact convex subsets in some Hausdorff topological vector spaces
 and morphisms are continuous affine maps. Note that for any $A\in EA$, $\Sigma(A)$ is a convex subset in $[0,1]^A\subset \mathbb R^A$, closed in the product topology, hence compact. Moreover, for any $f:A\to A'$, $\Sigma(f):\Sigma(A')\to \Sigma(A)$ is continuous. It follows that $\Sigma$ defines a functor $EA\to  CConv^{op}$. To avoid confusion, the functor in this case will be denoted by $\Sigma_c$. Let also $E_c:CConv^{op}\to EA$ be defined similarly as $E$, but now we also require the effect to be continuous. Put $T_c=E_c\circ\Sigma_c$.


It is well known that for any $K\in CConv$, $\Sigma_cE_c(K)\simeq K$, this induces a natural isomorphism $\mu^c:T_c^2\implies T_c$. Together with the unit $\eta: id\to T_c$, given by the evaluation map, $(T_c,\eta_c,\mu_c)$ defines a monad. Since $\mu$ is a natural isomorphism, this monad is idempotent and consequently, all algebras are isomorphisms. Hence if
$A\in EA^{T_c}$, $A\simeq T_c(A)$, so that $A$ is a complete archimedean. Conversely, if $A$ is complete archimedean, 
 then $a\mapsto ev_a$ establishes an isomorphism $A\simeq T_c(A)$. 


Let us return to  $\Sigma: EA\to BConv^{op}$, $E:BConv^{op}\to EA$.

\begin{lemma}\label{lemma:bas} Let $K\in BConv$. Then $VEK$ is an order unit Banach space with predual $V_bK$ and for any $f: K_1\to K_2$,
 $VEf=f^*$, where $f^*$ is the adjoint map of the extension  $f: V_bK_1\to V_bK_2$.

\end{lemma}

\begin{proof} Note that $VEK$ is the space $A_b(K)$ of bounded affine functions $K\to \mathbb R$. This is clearly an order unit Banach space, where the order unit norm is given by $\|f\|=\sup_K|f(x)|$. It is clear that any $f\in A_b(K)$ 
extends to a linear functional on $V_bK$ and for $v=\lambda x-\mu y$ we have
\[
|f(v)|\le \lambda|f(x)|+\mu|f(y)|\le (\lambda+\mu)\|f\|.
\]
Taking the infimum over all expressions for $v$ we obtain that $f\in V_b(K)$. Conversely, any $\varphi\in V_b(K)^*$ defines a 
 bounded affine map over $K$.

\end{proof}

\section{Monadicity}

We want to prove that the adjunction is monadic, applying the monadicity theorem, see \cite{leinster}. For this, we have to draw some diagrams.

Let $K,L\in BConv$ and let $f,g:K\to L$ be an $E$-absolute coequalizer pair in $BCOnv^{op}$. This means that there is some $A\in EA$ and an arrow $q:E(L)\to A$ such that
\begin{equation}\label{eq:absolute}\tag{*}
\xymatrix{
E(K) \ar@<.5ex>[r]^{Ef} \ar@<-.5ex>[r]_{Eg} &E(L)\ar[r]^q & A}
\end{equation}
is an absolute coequalizer diagram. That is, applying any functor $F: EA\to \mathcal C$ to \eqref{eq:absolute} yields a coequalizer diagram in $\mathcal C$. We have to show that
\begin{enumerate}
\item[(a)] there is some $e: L\to L'$ in $BConv^{op}$ such that 
\begin{equation}
\xymatrix{
E(K) \ar@<.5ex>[r]^{Ef} \ar@<-.5ex>[r]_{Eg} &E(L)\ar[r]^{Ee} & E(L')}
\end{equation}
is a coequalizer in $EA$
\item[(b)] each $e$ as in (a) is a coequalizer of $f$ and $g$ in $BConv^{op}$.
\end{enumerate}

Note first that since $E(K)$ and $E(L)$ are complete archimedean, we have 
\[
\xymatrix{
E(K) \ar@<.5ex>[r]^{Ef} \ar@<-.5ex>[r]_{Eg} \ar@{<->}^{\simeq}[d]&E(L)\ar[r]^q \ar@{<->}^\simeq[d]& A\\
T_cE(K)\ar@<.5ex>[r]^{T_cEf} \ar@<-.5ex>[r]_{T_cEg} &T_cE(L)\ar[r]^{T_cq}& T_c(A) }
\]
and since there is a coequalizer diagram in both lines, we obtain an isomorphism $T_c(A)\simeq A$. This implies that 
 $A$ is complete archimedean as well.

Let us now apply the functor $V:EA\to OVSu$ and  obtain the absolute coequalizer diagram 
\begin{equation}\label{eq:absolute_aous}\tag{**}
\xymatrix{
VE(K) \ar@<.5ex>[r]^{f^*} \ar@<-.5ex>[r]_{g^*} &VE(L)\ar[r]^{Vq} & VA }
\end{equation}
where $f^*$, $g^*$ are as in Lemma \ref{lemma:bas}. Note that $VE(K)$, $VE(L)$ and $VA$ are order unit Banach spaces and $f^*,g^*$ and $Vq$ are bounded linear maps.
%Note also that since $AOUS$ is a full subcategory in $OVSu$,  \eqref{eq:absolute_aous}  is a coequalizer diagram also in the category $AOUS$.
 Since  \eqref{eq:absolute_aous} is an absolute coequalizer diagram, applying the forgetful functor $U: OVSu\to Vect$ we obtain a coequalizer diagram in $Vect$. It follows that there is an isomorphism $VA\simeq VE(L)|_{R(f^*-g^*)}$ in $Vect$ such that the diagram
\[
\xymatrix{VE(L)\ar[r]^{Vq} \ar[rd]^{q'}& VA \ar[d]^{\simeq} \\  & VE(L)|_{R(f^*-g^*)}} 
\] 
commutes (in $Vect$), here $q'$ is the quotient map. It follows that 
\[
R((f-g)^*)=R(f^*-g^*)=(Vq)^{-1}(0)
\]
 and since $Vq$ is continuous, $R((f-g)^*)$ is closed (in the order unit norm topology of $VE(L)$). By the closed range theorem, it follows that
\[ 
R((f-g)^*)=N(f-g)^\perp
\]
where $N(f-g)=\{x\in V_b(L), (f-g)(x)=0\}$.
 and
\[
\xymatrix{VE(L)\ar[r]^{Vq} \ar[rd]^{q'} \ar[rdd]_{e^*=VEe}& VA \ar[d]^{\simeq} \\  & VE(L)|_{Ker(f-g)^\perp}\ar[d]^\simeq \\
& \left[Ker(f-g)\right]^*}
\]
where $e: Ker(f-g)\to V_b(L)$ is the embedding. We need to show that $Ker(f-g)$ is positively generated, that is, 
$Ker(f-g)=V_b(L')$, where 
\[
L':=Ker(f-g)\cap L=\{x\in L, f(x)=g(x)\}.
\]




 
\end{document}

