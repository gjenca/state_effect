\documentclass[12pt]{article}
\usepackage{amsmath, amsfonts, amsthm}
\usepackage[all]{xy}
\newtheorem{lemma}{Lemma}
\newtheorem{thm}{Theorem}
\newtheorem{prop}{Proposition}
\newcommand{\<}{\langle}
\theoremstyle{remark}
\newtheorem{ex}{Example}
\newtheorem{rem}{Remark}
\def\>{\rangle}
\newcommand{\ct}[1]{\mathbf{#1}}

\begin{document}

\section{Abstract convex structures}

Abstract convex structures were introduced by Stone \cite{stone}, see also \cite{gudder}, and further studied by e.g. \cite{fritz, jacobs}. A convex structure is a set $K$ endowed with a ternary operation $<\cdot,\cdot,\cdot>: [0,1]\times K\times K\to K$, satisfying the following conditions
\begin{enumerate}
\item[(c1)] $<\lambda, p,q> =<1-\lambda,q,p>$ (commutativity)
\item[(c2)] $<0,p,q>=p$ (endpoint condition)
\item[(c3)]  $<\lambda,p,<\mu,q,r>>=<\lambda\mu,<\nu,p,q>,r>$ ($\lambda\mu\neq 1$), where $\nu=\lambda(1-\mu)(1-\lambda\mu)^{-1}$ (associativity)
\item[(c4)] $<\lambda,p,p>=p$ (idempotence)
\end{enumerate}

\begin{ex}\label{ex:geomeric}
Let $V$ be a real vector space and let $K\subseteq V$ be a convex subset. Put $<\lambda,p,q>=(1-\lambda)p+\lambda q$, then $(K,<,,>)$ is a convex structure.
\end{ex}

Let $K$ be any convex structure. For $x,y\in K$, put
\begin{align*}
\sigma(x,y)&:=\inf\{ 0\leq \lambda\leq 1:<\lambda,x,x_1>=<\lambda,y,y_1>, x_1,y_1\in K\}\\
\rho(p,q)&:=\frac{\sigma(p,q)}{1-\sigma(p,q)}
\end{align*}
Since $<\frac{1}{2},x,y>=<\frac{1}{2},y,x>$, we have $0\leq \sigma(x,y)\leq \frac{1}{2}$, hence $0\leq \rho(p,q)\leq 1$. Moreover, $\sigma$ and $\rho$ are (equivalent)  semimetrics on $K$, called the intrinsic semimetrics, \cite{gudder}. 

Let $(K,< , ,>_K)$ and $(L,<, ,>_L)$ be convex structures, then a map $f:K\to L$ is affine if it satisfies
\[
f(<\lambda,x,y>_K)=<\lambda,f(x),f(y)>_L,\qquad \forall x,y\in K,\ \lambda\in [0,1].
\]
We will denote the category of convex structures with affine maps by $\ct{Conv}$. It was proved in \cite{jacobs} that $\ct{Conv}$ is the Eilenberg-Moore category for the distribution monad. 


A convex structure is called geometric (or a convex set) if it is isomorphic to a convex structure as in Example \cite{ex:geomeric}, more precisely, if there is a 
monomorphism $\phi:K\to V$ for some real vector space $V$. Any such monomorphism is called a geometric representation of $K$. Note that $\ct{Conv}(K,L)$ is naturally a convex structure, which is geometric for all $K\in \ct{Conv}$ if and only if $Y$ is geometric. Elements of $\ct{Conv}(K,\mathbb R)$, resp. $\ct{Conv}(K,[0,1])$, will be called (affine) functionals, resp. effects, on $K$. 
$\ct{Conv}(K,\mathbb R)$ inherits the structure of a real vector space, this will be denoted by $A(K)$. Moreover, the effects on $K$ form an effect algebra (see below), which will be denoted by $E(K)$.
 
The proofs of the following characterizations of geometric convex sets can be found in \cite{stone,gudder,fritz}.

\begin{thm} Let $(K,<,,>)$ be a convex structure. Then the following are equivalent.
\begin{enumerate}
\item[(i)] the following cancellation property holds: $<\lambda,z,x>=<\lambda,z,y>$ with some $\lambda \in (0,1)$ and  $x,y,z\in K$  implies that $x=y$.
\item[(ii)] $A(K)$ separates points of $K$: $f(x)=f(y)$ for all $f\in A(K)$ implies $x=y$.
\item[(iii)] $K$ is geometric.
\end{enumerate}


\end{thm}

From part (ii) of the above Theorem, we get a geometric representation $\phi: K\to A(K)'$ in the algebraic dual of $A(K)$, given by
\[
\phi(x)(f)=f(x),\qquad f\in A(K),\ x\in K.
\]
Note that this representation has the property that the image $\phi(K)$  lies in the hyperplane $\{ \varphi\in A(K)',\ \varphi(1_K)=1\}$, which does not contain 0. 
We will further identify $K$ with its image $\phi(K)$. 

 \begin{rem} Let $\psi:X\to V$ be  any geometric representation. Then clearly $\psi(K)$ is isomorphic to $\phi(K)$ and we may always assume that $\psi(K)$ is separated from 0 by considering the representation  $\tilde \psi:X\to V\oplus \mathbb R$  defined by $\tilde \psi(x)=(\phi(x),1)$. Hence the following constructions do not depend from the choice of geometric representation of $K$.

\end{rem}

Put  $V(K):=\mathrm{span}\{K\}\subseteq A(K)'$, $V(K)^+:=\cup_{\lambda\ge 0} \lambda K$. Let $u_K\in V(K)'$ be given by the restriction of the functional $1_K\in A(K)\subseteq A(K)''$. Then $V(K)^+$ is a generating cone in $V(K)$ and $K$ is a base of $V(K)^+$. Define
\[
\|v\|_K=\inf\{a+b,\ v=ax-by, a,b\ge 0, x,y\in K\}.
\] 
Then $\|\cdot\|_K$ is a seminorm in $V(K)$, which coincides with $2\rho$ on $K$, moreover, the intrinsic semimetric $\rho$ is a metric if and only if $\|\cdot\|_K$ is a norm, \cite{gudder}.  

\begin{thm} Let $K$ be a geometric convex structure. Then the following are equivalent.
\begin{enumerate}
\item[(i)] The intrinsic semimetric  $\rho$ (or $\sigma$) is a metric.
\item[(ii)] $K$ is separated by effects: $f(x)=f(y)$ for all $f\in E(K)$ implies $x=y$.
\item[(iii)] The set $S=co(K\cup -K)$ is linearly bounded.

\end{enumerate}




\end{thm}



\begin{proof} As mentioned above, $\rho$ is a metric iff $\|\cdot\|_K$ is a norm, which is equivalent to (iii) by \cite{ellis}. Assume (iii) holds and let $x\ne y\in K$. Then there is a bounded linear functional $\varphi$ on the normed space $(V(K),\|\cdot\|_K)$ 
such that $\varphi(x)\ne \varphi(y)$, we may assume $\|\varphi\|_K^*\le 1$. Then 
\[
\sup_{x\in K} |\varphi(x)|\le \sup_{\|v\|_K\le 1} |\varphi(v)|=\|\varphi\|_K^*\le 1,
\]
it follows that $f=\frac12(\varphi+1)$ is an effect and $f(x)\ne f(y)$. 
Conversely, assume (ii) and let $(1-\lambda)x+\lambda x'=(1-\lambda)y+\lambda y'$ for some $\lambda\in (0,1)$. Then
\[
|f(x)-f(y)|=\frac{\lambda}{1-\lambda}|f(x')-f(y')|\le \frac{2\lambda}{1-\lambda}.
\]
Hence if $\sigma(x,y)=0$, then $f(x)=f(y)$, so that  $x=y$.


\end{proof}
 
 A geometric convex structure $K$ is  \emph{bounded} if it satisfies any of  the above conditions. If, moreover,  $K$ is complete with respect to the intrinsic norm, then $(V(K),\|\cdot\|_K)$ is a Banach space, \cite{gudder}. The full subcategory $\ct{CConv}$ of 
  convex structures satisfying these properties will be important in the sequel. Its objects will be called \emph{complete bounded convex structures}. 
The following result is immediate from what was said before.

\begin{thm} 
$\ct{CConv}$ is equivalent to the category of base-normed Banach spaces with closed bases and base-preserving morphisms.
\end{thm}


\end{document}

